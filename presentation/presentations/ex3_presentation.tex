\documentclass[xcolor=table]{beamer}
\usepackage[utf8]{inputenc}
\usepackage[T1]{fontenc}

\usepackage{import}
\subimport{settings/}{general.tex}
\subimport{settings/}{colors.tex}
\subimport{settings/}{figures.tex}
\subimport{settings/}{math.tex}
\subimport{settings/}{code.tex}
\subimport{settings/}{fonts.tex}
\subimport{settings/}{commands.tex}
\setbeamertemplate{navigation symbols}{
    \insertsectionnavigationsymbol\
    \insertsubsectionnavigationsymbol\
}

% Emoji support for LaTeX
\usepackage{emoji}
\setemojifont{Noto Color Emoji}

% Importing images from another path
\graphicspath{{./images/}}

% Theme ------------------------------------------------------------------------
\usetheme{Antibes}
\usecolortheme{dolphin}
\setbeamercolor{structure}{fg=blue}
% \setbeamercolor{item}{fg=black}

% Title ------------------------------------------------------------------------
\title{\bft{OpenMP Implementation\\ of the Mandelbrot Set}}
\author{Marco Tallone}
\date{May 13, 2024}

\titlegraphic{
	\includegraphics[width=0.8\textwidth]{SDIC-DSAI-logos.png}
}

% Document ---------------------------------------------------------------------
\begin{document}
\frame{\titlepage} % This creates a title slide

% SLides -----------------------------------------------------------------------

%-------------------------------------------------------------------------------
\section{Mathematical Background}

\begin{frame}
\frametitle{Mathematical Background}

\begin{block}{Mandelbrot Set}
	The Mandelbrot set is the set of complex numbers $c$ for which the function
	$f_c(z) = z^2 + c$ does not diverge when iterated from $z = 0$.
\end{block}

\vspace{0.5cm}

\only<1->{
$\rightarrow$ elements farther than $2$ from the origin $\Rightarrow$ considered divergent
}

\vspace{0.5cm}

\only<2->{
$\rightarrow$ $\forall c \in S \subseteq \mathbb{C}$ iterate $f_c(z)$ from $z=0$ until 
\begin{equation*}
	|z_n| > 2 \quad \text{or} \quad n \geq I_{\max}
\end{equation*}
where $S = [x_{\min},x_{\max}] \times [y_{\min}, y_{\max}]$ is a subset of the complex plane and $I_{\max}$ is the maximum number of iterations.
}

\end{frame}

%-------------------------------------------------------------------------------
\section{OpenMP Implementation}

\subsection{Mandelbrot Function}

\begin{frame}[fragile]
\frametitle{Mandelbrot Function Implementation}

For a single point $c \in \mathbb{C}$, the previous can be verified with:

\begin{code}{C}{Mandelbrot Function}{}{}
short int (*mandelbrot*)(const complex double c,
						         const int I_max) {
    complex double z = 0.0;
    unsigned short int i = 0;
		while (cabs(z) < 2.0 && i < I_max) {
        z = z * z + c;
        i++;
    }
    return i;
}
\end{code}

\end{frame}
%-------------------------------------------------------------------------------
\subsection{Main Loop}

\begin{frame}[t]
\frametitle{Main Loop}

\only<1-2>{
\begin{block}{}
\emoji{warning} \bft{Note}: each point (=pixel) can be \bft{computed independently}
\end{block}
}

\only<2>{
	\begin{equation*}
		\Downarrow
	\end{equation*}
	\begin{itemize}
		\item pixels stored in a matrix \cc{M} of \cc{short int}
		\item each thread computes \bft{one row} at a time independently
		\item when a thread finishes a row, a new one is assigned
	\end{itemize}
}

\end{frame}

\begin{frame}[fragile]
\frametitle{Main Loop}

% {
% \fontsize{7pt}{10pt}\selectfont

\begin{code}{C2}{Main Loop}{}{}
[*#pragma omp parallel for schedule(dynamic)*]
for (int i = 0; i < n_y; i++) {

	const <*complex*> double im_c = (y_l + i * delta_y) * I;

	for (int j = 0; j < n_x; j++) {

		<*complex*> double c = (x_l + j * delta_x) + im_c;

		M[i][j] = (*mandelbrot*)(c, I_max);

	}
}
\end{code}
% }

\end{frame}
%-------------------------------------------------------------------------------
\section{Performance Analysis}

\subsection{Strong Scalability}

\begin{frame}[t]
\frametitle{Strong Scalability Analysis}

\only<1->{
\begin{block}{Strong Scaling}
    Measuring how the execution time $t$ varies with the number of threads $P$ for a fixed total workload.
\end{block}
}

\begin{itemize}
	\item<2-> \bft{EPYC} node: $2$ sockets, $64$ cores per socket

		\vspace{0.5cm}

	\item<3-> \bft{Total workload constant}:
		\begin{itemize}
			\item fixed-size image of $10^3 \times 10^3$ pixels
			\item $I_{\max} = 65535$ maximum possible for \cc{short int}
		\end{itemize}

		\vspace{0.5cm}

	\item<4-> \bft{Total number of threads}: $1$ $\rightarrow$ $128$
\end{itemize}

\end{frame}
%-------------------------------------------------------------------------------
\subsection{Plots}

\begin{frame}[fragile]
\frametitle{Strong Scalability Plots}

    % \begin{columns}

        % \column{0.5\textwidth}

            % \begin{adjustbox}{max width=\textwidth}
                \includegraphics[width=0.9\textwidth]{mandelbrot_scaling.pdf}
            % \end{adjustbox}

        % \column{0.5\textwidth}
\end{frame}
\begin{frame}[fragile]
\frametitle{Strong Scalability Plots}

            % \begin{adjustbox}{max width=\textwidth}
                \includegraphics[width=0.9\textwidth]{mandelbrot_speedup.pdf}
            % \end{adjustbox}

    % \end{columns}

\end{frame}
%-------------------------------------------------------------------------------

% \section{References}

% \nocite{*}

% \begin{frame}{References}
% \footnotesize % or \small or \tiny
% \bibliographystyle{plain}
% \bibliography{bibliography3}
% \end{frame}
%-------------------------------------------------------------------------------

\begin{frame}{References}
\nocite{*}
\begin{columns}[T] % align columns
\begin{column}{.5\textwidth}
\footnotesize % or \small or \tiny
\bibliographystyle{plain}
\bibliography{bibliography3}
\end{column}
\begin{column}{.5\textwidth}
\includegraphics[width=\textwidth]{mandelbrot.png} % replace with your image path
\end{column}
\end{columns}
\end{frame}


\end{document}
