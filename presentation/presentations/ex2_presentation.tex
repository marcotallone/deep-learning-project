\documentclass[xcolor=table]{beamer}
\usepackage[utf8]{inputenc}
\usepackage[T1]{fontenc}

\usepackage{import}
\subimport{settings/}{general.tex}
\subimport{settings/}{colors.tex}
\subimport{settings/}{figures.tex}
\subimport{settings/}{math.tex}
\subimport{settings/}{code.tex}
\subimport{settings/}{fonts.tex}
\subimport{settings/}{commands.tex}
\setbeamertemplate{navigation symbols}{
    \insertsectionnavigationsymbol\
    \insertsubsectionnavigationsymbol\
}

% Emoji support for LaTeX
\usepackage{emoji}
\setemojifont{Noto Color Emoji}

% Importing images from another path
\graphicspath{{./images/}}


% Theme ------------------------------------------------------------------------
\usetheme{Antibes}
\usecolortheme{dolphin}
\setbeamercolor{structure}{fg=blue}
% \setbeamercolor{item}{fg=black}

% Title ------------------------------------------------------------------------
\title{\bft{Parallel Quicksort Algorithms\\ in MPI and OpenMP}}
\author{Marco Tallone}
\date{May 13, 2024}

\titlegraphic{
	\includegraphics[width=0.8\textwidth]{SDIC-DSAI-logos.png}
}

% Document ---------------------------------------------------------------------
\begin{document}
\frame{\titlepage} % This creates a title slide

% SLides -----------------------------------------------------------------------

%-------------------------------------------------------------------------------
\section{Introduction}
\begin{frame}
\frametitle{Introduction}

Different parallel versions of the Quicksort algorithm in MPI and OpenMP have been implemented and compared:

\begin{itemize}
\item \bft{Task Quicksort} (OpenMP only)
\item \bft{Simple Parallel Quicksort}
\item \bft{Hyperquicksort}
\item \bft{PSRS} (Parallel Sorting by Regular Sampling)
\end{itemize}

\end{frame}
%-------------------------------------------------------------------------------
\section{Quicksort Algorithms}
\subsection{Serial Quicksort}
\begin{frame}
\frametitle{Serial Quicksort}
\begin{columns}
\column{0.3\textwidth}
\begin{enumerate}
\item<1-> Choose a pivot $\tau$ from the input array $\mathbf{X}$;
\item<2-> Partition $X$ into two subarrays $\mathbf{X}_{< \tau}$ and $\mathbf{X}_{\geq \tau}$;
\item<3-> Recursively sort $\mathbf{X}_{< \tau}$ and $\mathbf{X}_{\geq \tau}$ until $\text{len}(\mathbf{X}) \leq 2$.
\end{enumerate}

\column{0.7\textwidth}
\begin{adjustbox}{max width=\textwidth}
    \begin{tikzpicture}

        % Define the colors
        \definecolor{lightblue}{HTML}{0080ff}

        % 1D array rectangle
        \draw[line width=0.2] (0, 0) rectangle (10, 0.5);

        % Highlight the pivot of the array in yellow
        \draw[fill=yellow, opacity=0.5] (4.5, 0) rectangle (5, 0.5);
        \node at (4.75, -0.25) {pivot $\tau$};

        % Fill the array with the numbers
        \foreach \value [count=\i] in {7, 13, 18, 2, 17, 1, 14, 20, 6, 10,
                                       15, 9, 3, 16, 19, 4, 11, 12, 5, 8} {
            \node at (\i*0.5-0.25, 0.25) {\value};
            \draw[line width=0.1] (\i*0.5, 0) -- (\i*0.5, 0.5);
        }

        % Partitioned array below
        \pause
        \draw[fill=lightblue, opacity=0.3] (0, -1.5) rectangle (4.5, -1);
        \draw[fill=yellow, opacity=0.5] (4.5, -1.5) rectangle (5, -1);
        \draw[fill=red, opacity=0.4] (5, -1.5) rectangle (10, -1);
        \draw[line width=1, dashed, color=black] (4.5, -2) -- (4.5, -0.5);
        \draw[line width=0.2] (0, -1.5) rectangle (10, -1);
        \foreach \value [count=\i] in {7, 2, 1, 6, 9, 3, 4, 5, 8, 10, 
                                       13, 18, 17, 14, 20, 15, 16, 19, 11, 12} {
            \node at (\i*0.5-0.25, -1.25) {\value};
            \draw[line width=0.1] (\i*0.5, -1.5) -- (\i*0.5, -1);
        }
        \node at (4.75, -1.75) {$\tau$};
        \node at (2.25, -1.75) {$\mathbf{X}_{< \tau}$};
        \node at (7.5, -1.75) {$\mathbf{X}_{\geq \tau}$};

        % Recursive steps
        \pause
        \draw[-latex, line width=1, color=black] (2.25, -2.25) -- (2, -3.25);
        \draw[-latex, line width=1, color=black] (7.5, -2.25) -- (7.75, -3.25);

        % Little left recursive step
        \draw[fill=lightblue, opacity=0.3] (0, -4) rectangle (2, -3.5);
        \draw[fill=yellow, opacity=0.5] (2, -4) rectangle (2.5, -3.5);
        \draw[fill=red, opacity=0.4] (2.5, -4) rectangle (4, -3.5);
        \draw[line width=0.2] (0, -4) rectangle (4, -3.5);
        \foreach \i in {1, 2, 3, 4, 5, 6, 7} {
            \draw[line width=0.1] (\i*0.5, -4) -- (\i*0.5, -3.5);
        }
        \draw[-latex, line width=1, color=black] (1.25, -4.25) -- (1, -5.25);
        \draw[-latex, line width=1, color=black] (2.75, -4.25) -- (3, -5.25);
        \node at (1, -5.5) {\huge{\dots}};
        \node at (3, -5.5) {\huge{\dots}};

        % Little right recursive step
        \draw[fill=lightblue, opacity=0.3] (5, -4) rectangle (7.5, -3.5);
        \draw[fill=yellow, opacity=0.5] (7.5, -4) rectangle (8, -3.5);
        \draw[fill=red, opacity=0.4] (8, -4) rectangle (10, -3.5);
        \draw[line width=0.2] (5, -4) rectangle (10, -3.5);
        \foreach \i in {1, 2, 3, 4, 5, 6, 7, 8, 9} {
            \draw[line width=0.1] (5+\i*0.5, -4) -- (5+\i*0.5, -3.5);
        }
        \draw[-latex, line width=1, color=black] (6.25, -4.25) -- (6, -5.25);
        \draw[-latex, line width=1, color=black] (8.5, -4.25) -- (8.75, -5.25);
        \node at (6, -5.5) {\huge{\dots}};
        \node at (8.75, -5.5) {\huge{\dots}};
    \end{tikzpicture}
\end{adjustbox}
\end{columns}
\end{frame}
%-------------------------------------------------------------------------------
\begin{frame}[fragile]

\frametitle{Serial Quicksort}

\begin{code}{Pseudocode}{Serial Quicksort}{}{}
function (*serial_qsort*)($\mathbf{X}$):
    if len($\mathbf{X}$) > 2:
        $\tau \leftarrow$ choose_pivot($\mathbf{X}$)
        $\mathbf{X}_{< \tau}, \mathbf{X}_{\geq \tau} \leftarrow$ partition($\mathbf{X}, \tau$)


        (*serial_qsort*)($\mathbf{X}_{< \tau}$)


        (*serial_qsort*)($\mathbf{X}_{\geq \tau}$)
    else:
        if len($\mathbf{X}$) == 2 and $\mathbf{X}[0] > \mathbf{X}[1]$:
            swap($\mathbf{X}[0], \mathbf{X}[1]$)
\end{code}

\end{frame}
%-------------------------------------------------------------------------------
\subsection{OpenMP Task Quicksort}
\begin{frame}[fragile]

\frametitle{Task Quicksort}

\begin{code}{Pseudocode}{OpenMP Task Quicksort}{}{}
function (*omp_task_qsort*)($\mathbf{X}$):
    if len($\mathbf{X}$) > 2:
        $\tau \leftarrow$ choose_pivot($\mathbf{X}$)
        $\mathbf{X}_{< \tau}, \mathbf{X}_{\geq \tau} \leftarrow$ partition($\mathbf{X}, \tau$)

        [*#pragma omp task*]
        (*omp_task_qsort*)($\mathbf{X}_{< \tau}$)

        [*#pragma omp task*]
        (*omp_task_qsort*)($\mathbf{X}_{\geq \tau}$)
    else:
        if len($\mathbf{X}$) == 2 and $\mathbf{X}[0] > \mathbf{X}[1]$:
            swap($\mathbf{X}[0], \mathbf{X}[1]$)
\end{code}

\tiny{\emoji{warning} This function needs to be called \bft{inside a} \cc{\#pragma omp parallel} \bft{parallel region} to work properly.}
\end{frame}
%-------------------------------------------------------------------------------
\subsection{Simple Parallel Quicksort}
\begin{frame}[fragile]

\frametitle{Simple Parallel Quicksort: MPI implementation}

\begin{columns}
\column{0.4\textwidth}
\begin{enumerate}
\item<1-> Split $\mathbf{X}$ in $P$ chunks;
\item<2-> Choose \bft{one} pivot $\tau$ and broadcast it;
\item<3-> Local partitioning in $\mathbf{X}^{i}_{< \tau}$ and $\mathbf{X}^{i}_{\geq \tau}$;
\item<4-> ``Low'' partitions $\rightarrow$~rank $< P/2$,\\
``High'' partitions $\rightarrow$~rank $\geq P/2$;
\item<5-> Recursively sort the chunks.
\end{enumerate}

\column{0.6\textwidth}
\begin{adjustbox}{max width=\textwidth}
    \begin{tikzpicture}
        % Define the colors
        \definecolor{lightblue}{HTML}{0080ff}

        % Initial array
        \coordinate (A0) at  (0, 0);
        \draw[line width=0.2] (A0) rectangle ++(10, 0.5);
        \foreach \value [count=\i] in {7, 13, 18, 2, 17, 1, 14, 20, 6, 10,
                                       15, 9, 3, 16, 19, 4, 11, 12, 5, 8} {
            \node at ([xshift=\i*0.5cm-0.25cm, yshift=0.25cm]A0) {\value};
            \draw[line width=0.1] ([xshift=\i*0.5cm]A0) -- ++(0, 0.5);
        }

        % Separation of the chunks
        \draw[line width=0.5, dashed, color=gray] 
        ([yshift=0.5cm]A0) -- ++(0, 1);
        \draw[line width=0.5, dashed, color=gray]
        ([xshift=2.5cm, yshift=0.5cm]A0) -- ++(0, 1);
        \draw[line width=0.5, dashed, color=gray]
        ([xshift=5cm, yshift=0.5cm]A0) -- ++(0, 1);
        \draw[line width=0.5, dashed, color=gray]
        ([xshift=7.5cm, yshift=0.5cm]A0) -- ++(0, 1);
        \draw[line width=0.5, dashed, color=gray]
        ([xshift=10cm, yshift=0.5cm]A0) -- ++(0, 1);
        \node at ([xshift=1.25cm, yshift=1cm]A0) {$P_0$};
        \node at ([xshift=3.75cm, yshift=1cm]A0) {$P_1$};
        \node at ([xshift=6.25cm, yshift=1cm]A0) {$P_2$};
        \node at ([xshift=8.75cm, yshift=1cm]A0) {$P_3$};

        % Highlight the first cell of first chunk in yellow
        \pause
        \draw[fill=yellow, opacity=0.5] (A0) rectangle ++(0.5, 0.5);
        \node at ([xshift=0.25cm, yshift=-0.25cm]A0) {pivot $\tau$};

        % Array after partitioning
        \pause
        \coordinate (A1) at ([yshift=-2cm]A0);
        \draw[line width=0.2] (A1) rectangle ++(10, 0.5);

        \draw[fill=lightblue, opacity=0.3]
        (A1) rectangle ++(0.5, 0.5);
        \draw[fill=red, opacity=0.3]
        ([xshift=0.5cm]A1) rectangle ++(2, 0.5);

        \draw[fill=lightblue, opacity=0.3]
        ([xshift=2.5cm]A1) rectangle ++(1, 0.5);
        \draw[fill=red, opacity=0.3]
        ([xshift=3.5cm]A1) rectangle ++(1.5, 0.5);

        \draw[fill=lightblue, opacity=0.3]
        ([xshift=5cm]A1) rectangle ++(0.5, 0.5);
        \draw[fill=red, opacity=0.3]
        ([xshift=5.5cm]A1) rectangle ++(2, 0.5);

        \draw[fill=lightblue, opacity=0.3]
        ([xshift=7.5cm]A1) rectangle ++(1, 0.5);
        \draw[fill=red, opacity=0.3]
        ([xshift=8.5cm]A1) rectangle ++(1.5, 0.5);

        \foreach \value [count=\i] in {
        2,7,13,18,17,
        1,6,14,20,10,
        3,15,9,16,19,
        4,5,11,12,8} {
            \node at ([xshift=\i*0.5cm-0.25cm, yshift=0.25cm]A1) {\value};
            \draw[line width=0.1] ([xshift=\i*0.5cm]A1) -- ++(0, 0.5);
        }
        \draw[line width=0.5, dashed, color=gray]
        ([yshift=0.5cm]A1) -- ++(0, 1);
        \draw[line width=0.5, dashed, color=gray]
        ([xshift=2.5cm, yshift=0.5cm]A1) -- ++(0, 1.5);
        \draw[line width=0.5, dashed, color=gray]
        ([xshift=5cm, yshift=0.5cm]A1) -- ++(0, 1.5);
        \draw[line width=0.5, dashed, color=gray]
        ([xshift=7.5cm, yshift=0.5cm]A1) -- ++(0, 1.5);
        \draw[line width=0.5, dashed, color=gray]
        ([xshift=10cm, yshift=0.5cm]A1) -- ++(0, 1.5);

        % Exchange of partitions
        \pause
        \coordinate (A2) at ([yshift=-2cm]A1);
        \draw[line width=0.2] (A2) rectangle ++(10, 0.5);

        \draw[fill=lightblue, opacity=0.3]
        (A2) rectangle ++(3, 0.5);
        \draw[fill=red, opacity=0.3]
        ([xshift=3cm]A2) rectangle ++(7, 0.5);

        \foreach \value [count=\i] in {
        2,3,
        1,6,4,5,
        7,13,18,17,15,9,16,19,
        14,20,10,11,12,8} {
            \node at ([xshift=\i*0.5cm-0.25cm, yshift=0.25cm]A2) {\value};
            \draw[line width=0.1] ([xshift=\i*0.5cm]A2) -- ++(0, 0.5);
        }
        \draw[line width=0.5, dashed, color=gray]
        (A1) -- (A2);
        \draw[line width=0.5, dashed, color=gray]
        ([xshift=2.5cm]A1) -- ([xshift=1cm, yshift=0.5cm]A2);
        \draw[line width=0.5, dashed, color=gray]
        ([xshift=5cm]A1) -- ([xshift=3cm, yshift=0.5cm]A2);
        \draw[line width=0.5, dashed, color=gray]
        ([xshift=7.5cm]A1) -- ([xshift=7cm, yshift=0.5cm]A2);
        \draw[line width=0.5, dashed, color=gray]
        ([xshift=10cm]A1) -- ([xshift=10cm, yshift=0.5cm]A2);

        % Recursive step
        \pause
        \coordinate (A3) at ([yshift=-0.5cm]A2);        
        \draw[line width=0.5, dashed, color=gray]
        ([xshift=3cm, yshift=0.5cm]A3) -- ++(0, -2);

        \draw[-latex, line width=1, color=black]
        ([xshift=1.5cm]A3) -- ++(0, -1) node[midway, left] {$P_0$ \& $P_1$};
        \node at ([xshift=1.5cm, yshift=-1.5cm]A3) {\huge{\dots}};
        \draw[-latex, line width=1, color=black]
        ([xshift=6.5cm]A3) -- ++(0, -1) node[midway, right] {$P_2$ \& $P_3$};
        \node at ([xshift=6.5cm, yshift=-1.5cm]A3) {\huge{\dots}};

    \end{tikzpicture}
\end{adjustbox}
\end{columns}

\end{frame}
%//TODO: add pseudocode as extra slide at the end
%-------------------------------------------------------------------------------
\begin{frame}[fragile]

\frametitle{Simple Parallel Quicksort: OpenMP implementation}

\begin{columns}
\column{0.3\textwidth}
{
\fontsize{7pt}{10pt}\selectfont
\begin{enumerate}
\item<1-> Split $\mathbf{X}$ in $P$ chunks;
\item<2-> Choose \bft{one} pivot $\tau$ (\bft{shared} variable);
\item<3-> Local partitioning in $\mathbf{X}^{i}_{< \tau}$ and $\mathbf{X}^{i}_{\geq \tau}$;
\item<4-> $\mathbf{L} \leftarrow$ \# elements in low partitions,\\
$\mathbf{H} \leftarrow$ \# elements in high partitions;
\item<5-> $\mathbf{L}$ and $\mathbf{H}$ prefix sums;
\item<6-> Update a \bft{shared} \cc{indexes} array;
\item<7-> \bft{\red{Serially reorder}} according to \cc{indexes};
\item<8-> Repeat recursively.
\end{enumerate}
}

\column{0.7\textwidth}
\begin{adjustbox}{max width=\textwidth}
    \begin{tikzpicture}
        % Define the colors
        \definecolor{lightblue}{HTML}{0080ff}

        % 1D array rectangle
        \draw[line width=0.2] (0, 0) rectangle (10, 0.5);

        % Fill the array with the numbers
        \foreach \value [count=\i] in {7, 13, 18, 2, 17, 1, 14, 20, 6, 10,
                                       15, 9, 3, 16, 19, 4, 11, 12, 5, 8} {
            \node at (\i*0.5-0.25, 0.25) {\value};
            \draw[line width=0.1] (\i*0.5, 0) -- (\i*0.5, 0.5);
        }

        % Separation of the chunks
        \draw[line width=0.5, dashed, color=gray] (0, 0.5) -- (0, 1);
        \node at (1, 0.75) {$P_0$};
        \draw[line width=0.5, dashed, color=gray] (2, 0.5) -- (2, 1);
        \node at (3, 0.75) {$P_1$};
        \draw[line width=0.5, dashed, color=gray] (2, 0.5) -- (2, 1);
        \node at (5, 0.75) {$P_2$};
        \draw[line width=0.5, dashed, color=gray] (4, 0.5) -- (4, 1);
        \node at (7, 0.75) {$P_3$};
        \draw[line width=0.5, dashed, color=gray] (6, 0.5) -- (6, 1);
        \node at (9, 0.75) {$P_4$};
        \draw[line width=0.5, dashed, color=gray] (8, 0.5) -- (8, 1);
        \draw[line width=0.5, dashed, color=gray] (10, 0.5) -- (10, 1);

        % Highlight the first cell of first chunk in yellow
        \pause
        \draw[fill=yellow, opacity=0.5] (0, 0) rectangle (0.5, 0.5);
        \node at (0.25, -0.25) {pivot $\tau$};


        % Array after partitioning
        \pause
        \draw[line width=0.2] (0, -1) rectangle (10, -0.5);
        \draw[fill=lightblue, opacity=0.3] (0, -1) rectangle (0.5, -0.5);
        \draw[fill=red, opacity=0.3] (0.5, -1) rectangle (2, -0.5);
        \draw[fill=lightblue, opacity=0.3] (2, -1) rectangle (2.5, -0.5);
        \draw[fill=red, opacity=0.3] (2.5, -1) rectangle (4, -0.5);
        \draw[fill=lightblue, opacity=0.3] (4, -1) rectangle (4.5, -0.5);
        \draw[fill=red, opacity=0.3] (4.5, -1) rectangle (6, -0.5);
        \draw[fill=lightblue, opacity=0.3] (6, -1) rectangle (7, -0.5);
        \draw[fill=red, opacity=0.3] (7, -1) rectangle (8, -0.5);
        \draw[fill=lightblue, opacity=0.3] (8, -1) rectangle (8.5, -0.5);
        \draw[fill=red, opacity=0.3] (8.5, -1) rectangle (10, -0.5);
        \foreach \value [count=\i] in {2,7,18,13,1,17,14,20,6,10,15,9,3,4,19,16,5,12,11,8} {
            \node at (\i*0.5-0.25, -0.75) {\value};
            \draw[line width=0.1] (\i*0.5, -1) -- (\i*0.5, -0.5);
        }
        \draw[line width=0.5, dashed, color=gray] (2, -0.5) -- (2, 0);
        \draw[line width=0.5, dashed, color=gray] (2, -0.5) -- (2, 0);
        \draw[line width=0.5, dashed, color=gray] (4, -0.5) -- (4, 0);
        \draw[line width=0.5, dashed, color=gray] (6, -0.5) -- (6, 0);
        \draw[line width=0.5, dashed, color=gray] (8, -0.5) -- (8, 0);
        \draw[line width=0.5, dashed, color=gray] (10, -0.5) -- (10, 0);
        % \node[anchor=west] at (10, -0.75) {Partitioning};

        % Prefix sum arrays
        \pause
        \node[anchor=east] at (0.5, -2.25) {$\mathbf{L}$};
        \draw[line width=0.2] (0.5, -2.5) rectangle (3.5, -2);
        \foreach \value [count=\i] in {0,1,1,1,2,1} {
            \node at (\i*0.5+0.25, -2.25) {\value};
            \draw[line width=0.1] (\i*0.5, -2.5) -- (\i*0.5, -2);
        }

        \node[anchor=west] at (9.5, -2.25) {$\mathbf{H}$};
        \draw[line width=0.2] (6.5, -2.5) rectangle (9.5, -2);
        \foreach \value [count=\i] in {0,3,3,3,2,3} {
            \node at (\i*0.5+6.25, -2.25) {\value};
            \draw[line width=0.1] (\i*0.5+6, -2.5) -- (\i*0.5+6, -2);
        }

        % Dashed arrows from partitions to prefix sum arrays
        \draw[line width=0.5, dashed, color=lightblue, 
              opacity=0.5, -Stealth] (0.25, -1.05) -- (1.25, -1.95);
        \draw[line width=0.5, dashed, color=lightblue, 
              opacity=0.5, -Stealth] (2.25, -1.05) -- (1.75, -1.95);
        \draw[line width=0.5, dashed, color=lightblue, 
              opacity=0.5, -Stealth] (4.25, -1.05) -- (2.25, -1.95);
        \draw[line width=0.5, dashed, color=lightblue, 
              opacity=0.5, -Stealth] (6.5, -1.05) -- (2.75, -1.95);
        \draw[line width=0.5, dashed, color=lightblue, 
              opacity=0.5, -Stealth] (8.25, -1.05) -- (3.25, -1.95);
        \draw[line width=0.5, dashed, color=red, 
              opacity=0.5, -Stealth] (1.25, -1.05) -- (7.25, -1.95);
        \draw[line width=0.5, dashed, color=red, 
              opacity=0.5, -Stealth] (3.25, -1.05) -- (7.75, -1.95);
        \draw[line width=0.5, dashed, color=red, 
              opacity=0.5, -Stealth] (5.25, -1.05) -- (8.25, -1.95);
        \draw[line width=0.5, dashed, color=red, 
              opacity=0.5, -Stealth] (7.5, -1.05) -- (8.75, -1.95);
        \draw[line width=0.5, dashed, color=red, 
              opacity=0.5, -Stealth] (9.25, -1.05) -- (9.25, -1.95);

        \pause
        \node[anchor=west] at (1.1, -2.75) {prefix sum};
        \node[anchor=west] at (7.1, -2.75) {prefix sum};
        \draw[line width=0.5, color=gray, -Stealth] (0.75, -2.55) -- (0.75, -2.95);
        \draw[line width=0.5, color=gray, -Stealth] (3.25, -2.55) -- (3.25, -2.95);
        \draw[line width=0.5, color=gray, -Stealth] (6.75, -2.55) -- (6.75, -2.95);
        \draw[line width=0.5, color=gray, -Stealth] (9.25, -2.55) -- (9.25, -2.95);

        % Prefix sum arrays n2
        \node[anchor=east] at (0.5, -3.25) {$\mathbf{L}$};
        \draw[line width=0.2] (0.5, -3.5) rectangle (3.5, -3);
        \foreach \value [count=\i] in {0,1,2,3,5,6} {
            \node at (\i*0.5+0.25, -3.25) {\value};
            \draw[line width=0.1] (\i*0.5, -3.5) -- (\i*0.5, -3);
        }

        \node[anchor=west] at (9.5, -3.25) {$\mathbf{H}$};
        \draw[line width=0.2] (6.5, -3.5) rectangle (9.5, -3);
        \foreach \value [count=\i] in {0,3,6,9,11,14} {
            \node at (\i*0.5+6.25, -3.25) {\value};
            \draw[line width=0.1] (\i*0.5+6, -3.5) -- (\i*0.5+6, -3);
        }

        % Connecting lines
        \pause
        \draw[line width=0.5] (6.75, -1.05) -- (6.75, -1.5);
        \draw[line width=0.5] (6.75, -1.5) -- (4.5, -1.5);
        \draw[line width=0.5] (4.5, -1.5) -- (4.5, -4);
        \draw[line width=0.2, color=lightblue, fill=white] (4, -2.75) rectangle (5, -3.25);
        \node[anchor=center] at (4.5, -3) {$j=1$};
        \draw[line width=0.5, -Stealth] (2.25, -3.55) -- (2.25, -4.45);
        \node[anchor=east] at (2.25, -4) {$L[3] + 1 = 4$};
        \draw[fill=lightblue, opacity=0.15] (2.05, -4) circle (0.2);
        \draw[fill=lightblue, opacity=0.15] (2.25, -5.28) circle (0.2);
        \draw[line width=0.5] (2.25, -4) -- (4.5, -4);

        \draw[line width=0.5] (5.75, -1.05) -- (5.75, -4);
        \draw[line width=0.2, color=red, fill=white] (5.25, -2.75) rectangle (6.25, -2.25);
        \node[anchor=center] at (5.75, -2.5) {$j=2$};
        \draw[line width=0.5] (7.75, -3.55) -- (7.75, -4);
        \draw[line width=0.5] (5.75, -4) -- (7.75, -4);
        \draw[line width=0.5, -Stealth] (7.25, -4) -- (7.25, -4.45);
        \node[anchor=west] at (7.75, -4) {$L[5] + H[2] + 2 = 14$};
        \draw[fill=red, opacity=0.15] (11, -4) circle (0.2);
        \draw[fill=red, opacity=0.15] (7.25, -5.28) circle (0.2);

        \foreach \value [count=\i] in {0,1,2,3,4,5,6,7,8,9,10,11,12,13,14,15,16,17,18,19} {
            \node[color=gray] at (\i*0.5-0.25, -5.25) {\small\value};
        }

        % Final array
        \pause
        \draw[line width=0.2] (0, -5) rectangle (10, -4.5);
        \draw[fill=lightblue, opacity=0.3] (0, -5) rectangle (3, -4.5);
        \draw[fill=red, opacity=0.3] (3, -5) rectangle (10, -4.5);
        \foreach \value [count=\i] in {2,1,6,3,4,5,7,18,13,17,14,20,10,15,9,19,16,12,11,8} {
            \node at (\i*0.5-0.25, -4.75) {\value};
            \draw[line width=0.1] (\i*0.5, -5) -- (\i*0.5, -4.5);
        }

        % Recursive step
        \pause
        \draw[line width=0.5, dashed, color=gray] (3, -5) -- (3, -7);
        \draw[-latex, line width=1, color=black] (1.5, -5.5) -- ++(0, -1);
        \node at (1.5, -7) {\huge{\dots}};
        \draw[-latex, line width=1, color=black] (6.5, -5.5) -- ++(0, -1);
        \node at (6.5, -7) {\huge{\dots}};

    \end{tikzpicture}
\end{adjustbox}
\end{columns}

\end{frame}
%%-------------------------------------------------------------------------------
%\subsection{Hyperquicksort}
%\begin{frame}[fragile]

%\frametitle{Hyperquicksort}

%\begin{enumerate}
%\item<1-> Split $\mathbf{X}$ in $P$ chunks;
%\item<2-> \bft{\blue{Each process locally sorts its chunk}}
%\item<3-> \bft{\blue{One process chooses the median of its chunk as pivot $\tau$}} and broadcasts it;
%\item<3-> Local partitioning in $\mathbf{X}^{i}_{< \tau}$ and $\mathbf{X}^{i}_{\geq \tau}$;
%\item<4-> ``Low'' partitions $\rightarrow$~rank $< P/2$,\\
%``High'' partitions $\rightarrow$~rank $\geq P/2$;
%\item<5-> Recursively sort the chunks.
%\end{enumerate}

%\end{frame}
%%-------------------------------------------------------------------------------
%-------------------------------------------------------------------------------
\subsection{Hyperquicksort}
\begin{frame}[fragile]

\frametitle{Hyperquicksort}

\begin{enumerate}
\item Split $\mathbf{X}$ in $P$ chunks;
\item \bft{\blue{Each process locally sorts its chunk}}
\item \bft{\blue{One process chooses the median of its chunk as pivot $\tau$}} and broadcasts it;
\item Local partitioning in $\mathbf{X}^{i}_{< \tau}$ and $\mathbf{X}^{i}_{\geq \tau}$;
\item ``Low'' partitions $\rightarrow$~rank $< P/2$,\\
``High'' partitions $\rightarrow$~rank $\geq P/2$;
\item Recursively sort the chunks.
\end{enumerate}

\end{frame}
%-------------------------------------------------------------------------------

\subsection{PSRS}
\begin{frame}[fragile]

\frametitle{PSRS:\ MPI Implementation}

\begin{columns}
\column{0.4\textwidth}
{
\fontsize{7pt}{10pt}\selectfont
\begin{enumerate}
\item<1-> Split $\mathbf{X}$ in $P$ chunks;
\item<2-> Local chunk \bft{sorting};
\item<3-> Each selects $P$ samples:
\begin{equation*}
    \frac{i \cdot n}{P^2}\;,\; i = 0,1,\dots,P-1
\end{equation*}
\item<4-> One process:\\
- Gathers and sorts all samples;\\
- Regularly picks $P-1$ pivots;\\
- Broadcasts pivots.\\
\item<5-> Local partition according to pivots;
\item<6-> \bft{All-to-all} exchange:\\
$P_i$ sends $j^{th}$ partition to $P_j$;
\item <7-> Local sorting.
\end{enumerate}
}

\column{0.6\textwidth}
\begin{adjustbox}{max width=\textwidth}

    \begin{tikzpicture}

        % Define the colors
        \definecolor{lightblue}{HTML}{0080ff}

        % 1D array rectangle
        \coordinate (A0) at  (0, 0);
        \draw[line width=0.2] (A0) rectangle ++(10, 0.5);
        \foreach \value [count=\i] in {7, 13, 18, 2, 17, 1, 14, 20, 6, 10,
                                       15, 9, 3, 16, 19, 4, 11, 12, 5, 8} {
            \node at ([xshift=\i*0.5cm-0.25cm, yshift=0.25cm]A0) {\value};
            \draw[line width=0.1] ([xshift=\i*0.5cm]A0) -- ++(0, 0.5);
        }
        \draw[line width=0.5, dashed, color=gray]
        ([yshift=-0.5cm]A0) -- ++(0, 2);
        \draw[line width=0.5, dashed, color=gray]
        ([xshift=3.5cm, yshift=-0.5cm]A0) -- ++(0, 2);
        \draw[line width=0.5, dashed, color=gray]
        ([xshift=7cm, yshift=-0.5cm]A0) -- ++(0, 2);
        \draw[line width=0.5, dashed, color=gray]
        ([xshift=10cm, yshift=-0.5cm]A0) -- ++(0, 2);
        \node at ([xshift=1.75cm, yshift=1cm]A0) {$P_0$};
        \node at ([xshift=5.25cm, yshift=1cm]A0) {$P_1$};
        \node at ([xshift=8.5cm, yshift=1cm]A0) {$P_2$};

        % Local sorted sub-arrays
        \pause
        \coordinate (A1) at ([yshift=-1cm]A0);
        \draw[line width=0.2] (A1) rectangle ++(10, 0.5);
        \foreach \value [count=\i] in
        {1,2,7,13,14,17,18,3,6,9,10,15,16,20,4,5,8,11,12,19} {
            \node at ([xshift=\i*0.5cm-0.25cm, yshift=0.25cm]A1) {\value};
            \draw[line width=0.1] ([xshift=\i*0.5cm]A1) -- ++(0, 0.5);
        }
        \draw[line width=0.5, dashed, color=gray] (A1) -- ++(0, 0.5);
        \draw[line width=0.5, dashed, color=gray]
        ([xshift=3.5cm]A1) -- ++(0, 0.5);
        \draw[line width=0.5, dashed, color=gray]
        ([xshift=7cm]A1) -- ++(0, 0.5);
        \draw[line width=0.5, dashed, color=gray]
        ([xshift=10cm]A1) -- ++(0, 0.5);
        
        % Highlight pivots
        \pause
        \draw[fill=orange, opacity=0.3] ([xshift=0cm]A1) rectangle ++(0.5, 0.5);
        \draw[fill=orange, opacity=0.3] ([xshift=1cm]A1) rectangle ++(0.5, 0.5);
        \draw[fill=orange, opacity=0.3] ([xshift=2cm]A1) rectangle ++(0.5, 0.5);
        \draw[fill=orange, opacity=0.3] ([xshift=3.5cm]A1) rectangle ++(0.5, 0.5);
        \draw[fill=orange, opacity=0.3] ([xshift=4.5cm]A1) rectangle ++(0.5, 0.5);
        \draw[fill=orange, opacity=0.3] ([xshift=5.5cm]A1) rectangle ++(0.5, 0.5);
        \draw[fill=orange, opacity=0.3] ([xshift=7cm]A1) rectangle ++(0.5, 0.5);
        \draw[fill=orange, opacity=0.3] ([xshift=8cm]A1) rectangle ++(0.5, 0.5);
        \draw[fill=orange, opacity=0.3] ([xshift=9cm]A1) rectangle ++(0.5, 0.5);

        \pause
        \node[anchor=west] at ([yshift=-0.25cm]A1) {\textcolor{orange!90!black}{\textbf{Regularly sampled pivots:} $\mathbf{\tau_0=7; \quad \tau_1=12}$}};

        % Partition after pivot selection
        \pause
        \coordinate (A2) at ([yshift=-1cm]A1);

        \draw[fill=lightblue, opacity=0.3]
        ([xshift=2.5cm, yshift=0.05cm]A2) rectangle ++(0.7, 0.4);
        \draw[fill=yellow, opacity=0.5]
        ([xshift=3.5cm, yshift=0.05cm]A2) rectangle ++(2.1, 0.4);
        \draw[fill=red, opacity=0.3]
        ([xshift=5.8cm, yshift=0.05cm]A2) rectangle ++(1.1, 0.4);

        \node[anchor=west] at ([yshift=0.25cm]A2)
        {$\Rightarrow$ partition in $ < 7, \quad 7 \leq * <12, \quad \geq 12$};

        \draw[fill=lightblue, opacity=0.3]
        ([yshift=-0.5cm]A2) rectangle ++(1, 0.5);
        \draw[fill=yellow, opacity=0.5]
        ([xshift=1cm, yshift=-0.5cm]A2) rectangle ++(0.5, 0.5);
        \draw[fill=red, opacity=0.3]
        ([xshift=1.5cm, yshift=-0.5cm]A2) rectangle ++(2, 0.5);
        \draw[fill=lightblue, opacity=0.3]
        ([xshift=3.5cm, yshift=-0.5cm]A2) rectangle ++(1, 0.5);
        \draw[fill=yellow, opacity=0.5]
        ([xshift=4.5cm, yshift=-0.5cm]A2) rectangle ++(1, 0.5);
        \draw[fill=red, opacity=0.3]
        ([xshift=5.5cm, yshift=-0.5cm]A2) rectangle ++(1.5, 0.5);
        \draw[fill=lightblue, opacity=0.3]
        ([xshift=7cm, yshift=-0.5cm]A2) rectangle ++(1, 0.5);
        \draw[fill=yellow, opacity=0.5]
        ([xshift=8cm, yshift=-0.5cm]A2) rectangle ++(1, 0.5);
        \draw[fill=red, opacity=0.3]
        ([xshift=9cm, yshift=-0.5cm]A2) rectangle ++(1, 0.5);

        \draw[line width=0.2] ([yshift=-0.5cm]A2) rectangle ++(10, 0.5);
        \foreach \value [count=\i] in
        {1,2,7,13,14,17,18,3,6,9,10,15,16,20,4,5,8,11,12,19} {
            \node at ([xshift=\i*0.5cm-0.25cm, yshift=-0.25cm]A2) {\value};
            \draw[line width=0.1]
            ([xshift=\i*0.5cm, yshift=-0.5cm]A2) -- ++(0, 0.5);
        }
        \draw[line width=0.5, dashed, color=gray]
        (A2) -- ++(0, 1);
        \draw[line width=0.5, dashed, color=gray]
        ([xshift=3.5cm]A2) -- ++(0, 0.5);
        \draw[line width=0.5, dashed, color=gray]
        ([xshift=7cm]A2) -- ++(0, 0.5);
        \draw[line width=0.5, dashed, color=gray]
        ([xshift=10cm]A2) -- ++(0, 1);

        % \draw[line width=0.5, dashed, color=gray]
        % ([yshift=-0.5cm]A2) -- ++(0, -0.5);
        % \draw[line width=0.5, dashed, color=gray]
        % ([xshift=3.5cm, yshift=-0.5cm]A2) -- ++(0, -0.5);
        % \draw[line width=0.5, dashed, color=gray]
        % ([xshift=7cm, yshift=-0.5cm]A2) -- ++(0, -0.5);
        % \draw[line width=0.5, dashed, color=gray]
        % ([xshift=10cm, yshift=-0.5cm]A2) -- ++(0, -0.5);

        % All-to-all exchange
        \pause
        \coordinate (A3) at ([yshift=-2cm]A2);

        \draw[fill=lightblue, opacity=0.3]
        (A3) rectangle ++(3, 0.5);
        \draw[fill=yellow, opacity=0.5]
        ([xshift=3cm]A3) rectangle ++(2.5, 0.5);
        \draw[fill=red, opacity=0.3]
        ([xshift=5.5cm]A3) rectangle ++(4.5, 0.5);

        \draw[line width=0.2] (A3) rectangle ++(10, 0.5);
        \foreach \value [count=\i] in
        {1,2,3,6,4,5,7,9,10,8,11,13,14,17,18,15,16,20,12,19} {
            \node at ([xshift=\i*0.5cm-0.25cm, yshift=0.25cm]A3) {\value};
            \draw[line width=0.1] ([xshift=\i*0.5cm]A3) -- ++(0, 0.5);
        }
        \draw[line width=0.5, dashed, color=gray]
        ([yshift=0.5cm]A3) -- ([yshift=-0.5cm]A2);
        \draw[line width=0.5, dashed, color=gray]
        ([xshift=3cm, yshift=0.5cm]A3) -- ([xshift=3.5cm, yshift=-0.5cm]A2);
        \draw[line width=0.5, dashed, color=gray]
        ([xshift=5.5cm, yshift=0.5cm]A3) -- ([xshift=7cm, yshift=-0.5cm]A2);
        \draw[line width=0.5, dashed, color=gray]
        ([xshift=10cm, yshift=0.5cm]A3) -- ([xshift=10cm, yshift=-0.5cm]A2);

        % Local sorting
        \pause
        \coordinate (A4) at ([yshift=-2cm]A3);

        \draw[line width=0.2] (A4) rectangle ++(10, 0.5);
        \foreach \value [count=\i] in
        {1,2,3,4,5,6,7,8,9,10,11,12,13,14,15,16,17,18,19,20} {
            \node at ([xshift=\i*0.5cm-0.25cm, yshift=0.25cm]A4) {\value};
            \draw[line width=0.1] ([xshift=\i*0.5cm]A4) -- ++(0, 0.5);
        }
        \draw[line width=0.5, dashed, color=gray]
        ([yshift=0.5cm]A4) -- (A3);
        \draw[line width=0.5, dashed, color=gray]
        ([xshift=3cm, yshift=0.5cm]A4) -- ([xshift=3cm]A3);
        \draw[line width=0.5, dashed, color=gray]
        ([xshift=5.5cm, yshift=0.5cm]A4) -- ([xshift=5.5cm]A3);
        \draw[line width=0.5, dashed, color=gray]
        ([xshift=10cm, yshift=0.5cm]A4) -- ([xshift=10cm]A3);
        
    \end{tikzpicture}

\end{adjustbox}
\end{columns}
\end{frame}
%-------------------------------------------------------------------------------
\begin{frame}[fragile]

\frametitle{PSRS:\ OpenMP Implementation}

\begin{columns}
\column{0.4\textwidth}
{
\fontsize{7pt}{10pt}\selectfont
\begin{enumerate}
\item<1-> Split $\mathbf{X}$ in $P$ chunks;
\item<2-> Local chunk \bft{sorting};
\item<3-> Each selects $P$ samples:
$\rightarrow$~\bft{shared} variable;
\item<4-> Master thread:\\
- Sorts all samples;\\
- Regularly picks $P-1$ pivots;\\
- Writes pivots in shared array.\\
\item<5-> Local partition according to pivots;
\item<6-> $(P + 1) \times (P + 1)$ matrix $\mathbf{M}$;
\item<8-> Prefix sum $\mathbf{S}$ of last column;
\item<9-> Update the \cc{indexes} array;
\item<10-> \bft{\red{Serially reorder}} elements according to \cc{indexes};\\
\item<11-> Each thread sorts from\\
$\mathbf{S}[i]$ to $\mathbf{M}[i+1][P]$
\end{enumerate}
}

\column{0.6\textwidth}
\begin{adjustbox}{max width=\textwidth}

    \begin{tikzpicture}

        % Define the colors
        \definecolor{lightblue}{HTML}{0080ff}

        % 1D array rectangle
        \coordinate (A0) at  (0, 0);
        \draw[line width=0.2] (A0) rectangle ++(10, 0.5);
        \foreach \value [count=\i] in {7, 13, 18, 2, 17, 1, 14, 20, 6, 10,
                                       15, 9, 3, 16, 19, 4, 11, 12, 5, 8} {
            \node at ([xshift=\i*0.5cm-0.25cm, yshift=0.25cm]A0) {\value};
            \draw[line width=0.1] ([xshift=\i*0.5cm]A0) -- ++(0, 0.5);
        }
        \draw[line width=0.5, dashed, color=gray]
        ([yshift=0.5cm]A0) -- ++(0, 0.5);
        \draw[line width=0.5, dashed, color=gray]
        ([xshift=3.5cm, yshift=0.5cm]A0) -- ++(0, 0.5);
        \draw[line width=0.5, dashed, color=gray]
        ([xshift=7cm, yshift=0.5cm]A0) -- ++(0, 0.5);
        \draw[line width=0.5, dashed, color=gray]
        ([xshift=10cm, yshift=0.5cm]A0) -- ++(0, 0.5);
        \node at ([xshift=1.75cm, yshift=0.75cm]A0) {$P_0$};
        \node at ([xshift=5.25cm, yshift=0.75cm]A0) {$P_1$};
        \node at ([xshift=8.5cm, yshift=0.75cm]A0) {$P_2$};

        % Local sorted sub-arrays
        \pause
        \draw[line width=0.2] (0, -1) rectangle (10, -0.5);
        \foreach \value [count=\i] in {1,2,7,13,14,17,18,3,6,9,10,15,16,20,4,5,8,11,12,19} {
            \node at (\i*0.5-0.25, -0.75) {\value};
            \draw[line width=0.1] (\i*0.5, -1) -- (\i*0.5, -0.5);
        }
        \draw[line width=0.5, dashed, color=gray] (0, -0.5) -- (0, 0);
        \draw[line width=0.5, dashed, color=gray] (3.5, -0.5) -- (3.5, 0);
        \draw[line width=0.5, dashed, color=gray] (7, -0.5) -- (7, 0);
        \draw[line width=0.5, dashed, color=gray] (10, -0.5) -- (10, 0);
        
        % Highlight pivots
        \pause
        \draw[fill=orange, opacity=0.3] (0, -1) rectangle (0.5, -0.5);
        \draw[fill=orange, opacity=0.3] (1, -1) rectangle (1.5, -0.5);
        \draw[fill=orange, opacity=0.3] (2, -1) rectangle (2.5, -0.5);
        \draw[fill=orange, opacity=0.3] (3.5, -1) rectangle (4, -0.5);
        \draw[fill=orange, opacity=0.3] (4.5, -1) rectangle (5, -0.5);
        \draw[fill=orange, opacity=0.3] (5.5, -1) rectangle (6, -0.5);
        \draw[fill=orange, opacity=0.3] (7, -1) rectangle (7.5, -0.5);
        \draw[fill=orange, opacity=0.3] (8, -1) rectangle (8.5, -0.5);
        \draw[fill=orange, opacity=0.3] (9, -1) rectangle (9.5, -0.5);

        \pause
        \node[anchor=west] at (0, -1.25) {\textcolor{orange!90!black}{\textbf{Regularly sampled pivots:} $\mathbf{\tau_0=7; \quad \tau_1=12}$}};

        % Partition after pivot selection
        \pause
        \draw[fill=lightblue, opacity=0.3] (2.5, -1.95) rectangle (3.2, -1.55);
        \draw[fill=yellow, opacity=0.5] (3.5, -1.95) rectangle (5.6, -1.55);
        \draw[fill=red, opacity=0.3] (5.8, -1.95) rectangle (6.9, -1.55);
        \node[anchor=west] at (0, -1.75)
        {$\Rightarrow$ partition in $ < 7, \quad 7 \leq * <12, \quad \geq 12$};

        \draw[fill=lightblue, opacity=0.3] (0, -2.5) rectangle (1, -2);
        \draw[fill=yellow, opacity=0.5] (1, -2.5) rectangle (1.5, -2);
        \draw[fill=red, opacity=0.3] (1.5, -2.5) rectangle (3.5, -2);
        \draw[fill=lightblue, opacity=0.3] (3.5, -2.5) rectangle (4.5, -2);
        \draw[fill=yellow, opacity=0.5] (4.5, -2.5) rectangle (5.5, -2);
        \draw[fill=red, opacity=0.3] (5.5, -2.5) rectangle (7, -2);
        \draw[fill=lightblue, opacity=0.3] (7, -2.5) rectangle (8, -2);
        \draw[fill=yellow, opacity=0.5] (8, -2.5) rectangle (9, -2);
        \draw[fill=red, opacity=0.3] (9, -2.5) rectangle (10, -2);

        \draw[line width=0.2] (0, -2.5) rectangle (10, -2);
        \foreach \value [count=\i] in {1,2,7,13,14,17,18,3,6,9,10,15,16,20,4,5,8,11,12,19} {
            \node at (\i*0.5-0.25, -2.25) {\value};
            \draw[line width=0.1] (\i*0.5, -2.5) -- (\i*0.5, -2);
        }
        \draw[line width=0.5, dashed, color=gray] (0, -2) -- (0, -1);
        \draw[line width=0.5, dashed, color=gray] (3.5, -1.5) -- (3.5, -1);
        \draw[line width=0.5, dashed, color=gray] (7, -2) -- (7, -1.5);
        \draw[line width=0.5, dashed, color=gray] (10, -2) -- (10, -1);

        \draw[line width=0.5, dashed, color=gray] (0, -2.5) -- (0, -3);
        \draw[line width=0.5, dashed, color=gray] (3.5, -2.5) -- (3.5, -3);
        \draw[line width=0.5, dashed, color=gray] (7, -2.5) -- (7, -3);
        \draw[line width=0.5, dashed, color=gray] (10, -2.5) -- (10, -3);

        %4x4 matrix filled with zeroes below
        \only<6>{
            \node[anchor=west] at (9.25, -4.5) {$\mathbf{M}$};
            \draw[fill=lightblue, opacity=0.3] (7.5, -4) rectangle (9, -4.5);
            \draw[fill=yellow, opacity=0.5] (7.5, -4.5) rectangle (9, -5);
            \draw[fill=red, opacity=0.3] (7.5, -5) rectangle (9, -5.5);
            \draw[line width=0.2] (7, -3.5) rectangle (9, -5.5);
            \draw[line width=0.2] (7.5, -3.5) -- (7.5, -5.5);
            \draw[line width=0.2] (8, -3.5) -- (8, -5.5);
            \draw[line width=0.2] (8.5, -3.5) -- (8.5, -5.5);
            \draw[line width=0.2] (7, -4) -- (9, -4);
            \draw[line width=0.2] (7, -4.5) -- (9, -4.5);
            \draw[line width=0.2] (7, -5) -- (9, -5);

            \node at (7.25, -3.75) {0};
            \node at (7.75, -3.75) {0};
            \node at (8.25, -3.75) {0};
            \node at (8.75, -3.75) {0};

            \node at (7.25, -4.25) {0};
            \node at (7.75, -4.25) {2};
            \node at (8.25, -4.25) {2};
            \node at (8.75, -4.25) {2};

            \node at (7.25, -4.75) {0};
            \node at (7.75, -4.75) {1};
            \node at (8.25, -4.75) {2};
            \node at (8.75, -4.75) {2};

            \node at (7.25, -5.25) {0};
            \node at (7.75, -5.25) {4};
            \node at (8.25, -5.25) {3};
            \node at (8.75, -5.25) {2};
        }

        \only<7->{
            \node[anchor=west] at (9.25, -4.5) {$\mathbf{M}$};
            \draw[line width=0.2] (7, -3.5) rectangle (9, -5.5);
            \draw[line width=0.2] (7.5, -3.5) -- (7.5, -5.5);
            \draw[line width=0.2] (8, -3.5) -- (8, -5.5);
            \draw[line width=0.2] (8.5, -3.5) -- (8.5, -5.5);
            \draw[line width=0.2] (7, -4) -- (9, -4);
            \draw[line width=0.2] (7, -4.5) -- (9, -4.5);
            \draw[line width=0.2] (7, -5) -- (9, -5);

            \node at (7.25, -3.75) {0};
            \node at (7.75, -3.75) {0};
            \node at (8.25, -3.75) {0};
            \node at (8.75, -3.75) {0};

            \node at (7.25, -4.25) {0};
            \node at (7.75, -4.25) {2};
            \node at (8.25, -4.25) {4};
            \node at (8.75, -4.25) {6};

            \node at (7.25, -4.75) {0};
            \node at (7.75, -4.75) {1};
            \node at (8.25, -4.75) {3};
            \node at (8.75, -4.75) {5};

            \node at (7.25, -5.25) {0};
            \node at (7.75, -5.25) {7};
            \node at (8.25, -5.25) {4};
            \node at (8.75, -5.25) {9};
        }

        \only<7>{
            % Arrow 'prefix sum' under the matrix from left to right
            \draw[-latex, line width=1, color=black] (7.25, -5.75) -- (8.75, -5.75) node[midway, below] {prefix sum};
        }

        %Highlight fill in gray last column
        \only<8->{
        \draw[fill=gray, opacity=0.3] (8.5, -3.5) rectangle (9, -5.5);

        % S array below matrix
        \draw[line width=0.2] (7, -6.5) rectangle (9, -6);
        \draw[line width=0.2] (7.5, -6.5) -- (7.5, -6);
        \draw[line width=0.2] (8, -6.5) -- (8, -6);
        \draw[line width=0.2] (8.5, -6.5) -- (8.5, -6);
        \draw[line width=0.2] (7, -6) -- (9, -6);

        \draw[fill=gray, opacity=0.3] (7, -6) rectangle (9, -6.5);

        \node at (7.25, -6.25) {0};
        \node at (7.75, -6.25) {6};
        \node at (8.25, -6.25) {11};
        \node at (8.75, -6.25) {20};

        \node[anchor=west] at (9.25, -6.25) {$\mathbf{S}$};
        }

        \only<8>{
            % arrow 'prefix sum' to the right of the matrix
            \draw[-latex, line width=1, color=gray] (9.25, -3.5) -- (9.25, -5.5) node[pos=0.8, right] {prefix sum};
        }

        % Final array
        \only<9->{
        \draw[fill=yellow, opacity=0.25] (4.25, -7.75) circle (0.2);
        \foreach \value [count=\i] in {0,1,2,3,4,5,6,7,8,9,10,11,12,13,14,15,16,17,18,19} {
            \node[color=gray] at (\i*0.5-0.25, -7.75) {\small\value};
        }

        % Connection LINES
        \draw[line width=0.5] (5.25, -2.55) -- (5.25, -3);
        \draw[line width=0.5] (5.25, -3) -- (4.25, -3);
        \draw[line width=0.5, -Stealth] (4.25, -3) -- (4.25, -6.95);
        \draw[line width=0.2, color=black, fill=white] (2.65, -3.25) rectangle (5.80, -3.75);
        \node[anchor=center] at (4.25, -3.5) {$i=1,\;j=1,\;k=1$};

        \draw[fill=yellow, opacity=0.25] (4.035, -5.5) circle (0.2);
        \node[anchor=east] at (4.25, -5.5) {$S[k] + M[k+1][i] + j = 8$};

        \draw[line width=0.5, color=gray, opacity=0.5] (7.65, -4.75) -- (1.5, -4.75);
        \draw[line width=0.5, color=gray, opacity=0.5] (1.5, -4.75) -- (1.5, -5.25);
        \draw[line width=0.5, color=black, opacity=0.5] (7.65, -6.25) -- (0.5, -6.25);
        \draw[line width=0.5, color=black, opacity=0.5] (0.5, -6.25) -- (0.5, -5.8);
        }

        \only<10->{
        \draw[line width=0.2] (0, -7.5) rectangle (10, -7);
        \draw[fill=lightblue, opacity=0.3] (0, -7.5) rectangle (3, -7);
        \draw[fill=yellow, opacity=0.5] (3, -7.5) rectangle (5.5, -7);
        \draw[fill=red, opacity=0.3] (5.5, -7.5) rectangle (10, -7);
        \foreach \value [count=\i] in {1,2,3,6,4,5,7,9,10,8,11,13,14,17,18,15,16,20,12,19} {
            \node at (\i*0.5-0.25, -7.25) {\value};
            \draw[line width=0.1] (\i*0.5, -7.5) -- (\i*0.5, -7);
        }
        }

        \only<11->{
        \node at (1.5, -8.5) {$P_0$ sorts first};
        \node at (4.25, -8.5) {$P_1$ sorts second};
        \node at (7.5, -8.5) {$P_2$ sorts third};

        \draw[line width=0.2] (0, -9) rectangle (10, -9.5);
        \foreach \value [count=\i] in {1,2,3,4,5,6,7,8,9,10,
                                       11,12,13,14,15,16,17,18,19,20} {
            \node at (\i*0.5-0.25, -9.25) {\value};
            \draw[line width=0.1] (\i*0.5, -9) -- (\i*0.5, -9.5);
        }
        \draw[line width=0.5, dashed, color=gray] (0, -8) -- (0, -9);
        \draw[line width=0.5, dashed, color=gray] (3, -8) -- (3, -9);
        \draw[line width=0.5, dashed, color=gray] (5.5, -8) -- (5.5, -9);
        \draw[line width=0.5, dashed, color=gray] (10, -8) -- (10, -9);
        }

    \end{tikzpicture}

\end{adjustbox}
\end{columns}
\end{frame}
%-------------------------------------------------------------------------------
\section{Results}

\begin{frame}[t]
\frametitle{Scaling Analysis}

\begin{block}{Strong Scaling}
    Measuring how the execution time $t$ varies with the number of processors $P$ for a fixed total workload.
\end{block}

\begin{block}{Weak Scaling}
    Measuring how the execution time $t$ varies with the number of processors $P$ for a fixed workload per processor.
\end{block}

\begin{columns}

    \column{0.5\textwidth}
    
    \bft{MPI Implementation}
    \begin{itemize}
        \item \hyperlink{strong-scaling-time}{\emoji{chart-increasing} Strong Scaling $t - P$};
        \item \hyperlink{strong-scaling-speedup}{\emoji{chart-increasing} Strong Scaling $S_p - P$};
        \item \hyperlink{weak-scaling-time}{\emoji{chart-increasing} Weak Scaling $t - P$};
        \item \hyperlink{weak-scaling-efficiency}{\emoji{chart-increasing} Weak Scaling $\varepsilon - P$}.
    \end{itemize}

    \column{0.5\textwidth}

    \bft{OpenMP Implementation}
    \begin{itemize}
        \item \hyperlink{omp-strong-scaling-time}{\emoji{chart-decreasing} Strong Scaling $t - P$};
        \item \hyperlink{omp-strong-scaling-speedup}{\emoji{chart-decreasing} Strong Scaling $S_p - P$};
        \item \hyperlink{omp-weak-scaling-time}{\emoji{chart-decreasing} Weak Scaling $t - P$};
        \item \hyperlink{omp-weak-scaling-efficiency}{\emoji{chart-decreasing} Weak Scaling $\varepsilon - P$}.
    \end{itemize}

\end{columns}

\end{frame}

\begin{frame}[label=strong-scaling-time]
\includegraphics[width=\textwidth]{mpi_timeplot.pdf}
\end{frame}

\begin{frame}[label=strong-scaling-speedup]
\includegraphics[width=\textwidth]{mpi_speedplot.pdf}
\end{frame}

\begin{frame}[label=weak-scaling-time]
\includegraphics[width=\textwidth]{mpi_weaktimeplot.pdf}
\end{frame}

\begin{frame}[label=weak-scaling-efficiency]
\includegraphics[width=\textwidth]{mpi_effplot.pdf}
\end{frame}

\begin{frame}[label=omp-strong-scaling-time]
\includegraphics[width=\textwidth]{omp_timeplot.pdf}
\end{frame}

\begin{frame}[label=omp-strong-scaling-speedup]
\includegraphics[width=\textwidth]{omp_speedplot.pdf}
\end{frame}

\begin{frame}[label=omp-weak-scaling-time]
\includegraphics[width=\textwidth]{omp_weaktimeplot.pdf}
\end{frame}

\begin{frame}[label=omp-weak-scaling-efficiency]
\includegraphics[width=\textwidth]{omp_effplot.pdf}
\end{frame}

%-------------------------------------------------------------------------------

        


% \subsection{MPI}

% \begin{frame}[fragile]
%     \frametitle{MPI Strong Scaling}
%     \begin{columns}
%         \column{0.5\textwidth}
%             \begin{adjustbox}{max width=\textwidth}
%                 \includegraphics[width=1.1\textwidth]{mpi_timeplot.pdf}
%             \end{adjustbox}
%         \column{0.5\textwidth}
%             \begin{adjustbox}{max width=\textwidth}
%                 \includegraphics[width=1.1\textwidth]{mpi_speedplot.pdf}
%             \end{adjustbox}
%     \end{columns}
% \end{frame}

% \begin{frame}[fragile]
%     \frametitle{MPI Weak Scaling}
%     \begin{columns}
%         \column{0.5\textwidth}
%             \begin{adjustbox}{max width=\textwidth}
%                 \includegraphics[width=1.1\textwidth]{mpi_weaktimeplot.pdf}
%             \end{adjustbox}
%         \column{0.5\textwidth}
%             \begin{adjustbox}{max width=\textwidth}
%                 \includegraphics[width=1.1\textwidth]{mpi_effplot.pdf}
%             \end{adjustbox}
%     \end{columns}
% \end{frame}

% \subsection{OpenMP}

% \begin{frame}[fragile]
%     \frametitle{OpenMP Strong Scaling}
%     \begin{columns}
%         \column{0.5\textwidth}
%             \begin{adjustbox}{max width=\textwidth}
%                 \includegraphics[width=1.1\textwidth]{omp_timeplot.pdf}
%             \end{adjustbox}
%         \column{0.5\textwidth}
%             \begin{adjustbox}{max width=\textwidth}
%                 \includegraphics[width=1.1\textwidth]{omp_speedplot.pdf}
%             \end{adjustbox}
%     \end{columns}
% \end{frame}

% \begin{frame}[fragile]
%     \frametitle{OpenMP Weak Scaling}
%     \begin{columns}
%         \column{0.5\textwidth}
%             \begin{adjustbox}{max width=\textwidth}
%                 \includegraphics[width=1.1\textwidth]{omp_weaktimeplot.pdf}
%             \end{adjustbox}
%         \column{0.5\textwidth}
%             \begin{adjustbox}{max width=\textwidth}
%                 \includegraphics[width=1.1\textwidth]{omp_effplot.pdf}
%             \end{adjustbox}
%     \end{columns}
% \end{frame}
%-------------------------------------------------------------------------------

\section{References}

\nocite{*}

\begin{frame}{References}
\footnotesize % or \small or \tiny
\bibliographystyle{plain}
\bibliography{bibliography2}
\end{frame}

\end{document}
