% Emoji support for LaTeX
\usepackage{emoji}
\setemojifont{Noto Color Emoji}

% Fontawesome5 package for icons
\usepackage{fontawesome5}

% tcolorbox for colored boxes
% [breakable] option allows the box to be split across pages
\usepackage[breakable]{tcolorbox}

% Save the original \fboxsep
\newlength{\originalfboxsep}

% --------------------------------- CALLOUTS -----------------------------------

% Warning block (and color variants) -------------------------------------------
\newtcolorbox{warning}{
    breakable,              % allow the box to be split across pages
    colback=boxcolor,       % background color
    colframe=yellow,        % frame color
    coltext=textcolor,      % text color
    boxrule=0.3mm,          % frame thickness
    left=0mm,               % left margin
    right=7mm,              % right margin
    top=3mm,                % top margin
    bottom=3mm,             % bottom margin
    enhanced jigsaw,        % better frame drawing
    overlay={               % overlay warning emoji
        \node[anchor=north west, 
              yshift=-2mm, 
              xshift=1mm] at (frame.north west) {\emoji{warning}};
    },
    left=7mm                % indent
}

\newtcolorbox{redwarning}{
    breakable,              % allow the box to be split across pages
    colback=darkred,        % background color
    colframe=darkred,       % frame color
    opacityback=0.5,        % background opacity
    opacityframe=0.5,       % frame opacity
    coltext=textcolor,      % text color
    boxrule=0mm,            % frame thickness
    left=0mm,               % left margin
    right=7mm,              % right margin
    top=3mm,                % top margin
    bottom=3mm,             % bottom margin
    enhanced jigsaw,        % better frame drawing
    overlay={               % overlay warning emoji
        \node[anchor=north west, 
              yshift=-2mm, 
              xshift=1mm] at (frame.north west) {\emoji{warning}};
    },
    left=7mm                % indent
}

% Info block (and color variants) ----------------------------------------------
\newtcolorbox{info}{
    breakable,              % allow the box to be split across pages
    colback=boxcolor,       % background color
    colframe=boxcolor,      % frame color
    coltext=textcolor,      % text color
    boxrule=0mm,            % frame thickness
    left=0mm,               % left margin
    right=7mm,              % right margin
    top=3mm,                % top margin
    bottom=3mm,             % bottom margin
    enhanced jigsaw,        % better frame drawing
    overlay={               % overlay drawing
        \node[anchor=north west, 
              yshift=-2.5mm, 
              xshift=1mm] at (frame.north west) 
              {\textcolor{iconscolor}{\faIcon{info-circle}}};
    },
    left=7mm                % indent
}

\newtcolorbox{blueinfo}{
    breakable,              % allow the box to be split across pages
    colback=darkblue,       % background color
    colframe=darkblue,      % frame color
    opacityback=0.3,        % background opacity
    opacityframe=0.3,       % frame opacity
    coltext=textcolor,      % text color
    boxrule=0mm,            % frame thickness
    left=0mm,               % left margin
    right=7mm,              % right margin
    top=3mm,                % top margin
    bottom=3mm,             % bottom margin
    enhanced jigsaw,        % better frame drawing
    overlay={               % overlay drawing
        \node[anchor=north west, 
              yshift=-2.5mm, 
              xshift=1mm] at (frame.north west) 
              {\textcolor{iconscolor}{\faIcon{info-circle}}};
    },
    left=7mm                % indent
}

% Definition block -------------------------------------------------------------
% Definition box small icon and counter
\newcommand{\defSquare}{
    \setlength{\originalfboxsep}{\fboxsep} 
    \setlength{\fboxsep}{2pt} % add padding
    \colorbox{iconscolor}{\textcolor{boxcolor}{\small\textbf{\textit{def} \thedefinition}}} % print the box
    \setlength{\fboxsep}{\originalfboxsep} % restore \fboxsep
}

% "def" block
\newcounter{definition}
\makeatletter
\@addtoreset{definition}{section}
\makeatother
\renewcommand{\thedefinition}{\thesection.\arabic{definition}}
\newtcolorbox{definition}[3][]{
    breakable,              % allow the box to be split across pages
    colback=boxcolor,       % background color
    colframe=boxcolor,      % frame color
    coltext=textcolor,      % text color
    boxrule=0mm,            % frame thickness
    left=0mm,               % left margin
    right=7mm,              % right margin
    top=3mm,                % top margin
    bottom=3mm,             % bottom margin
    enhanced jigsaw,        % better frame drawing
    overlay={               % overlay drawing
        \node[anchor=north west, 
              yshift=-2.9mm, 
              xshift=1mm] at (frame.north west) 
              {\textcolor{iconscolor}{\faIcon{book-open}}};
    },
    left=7mm,               % indent
    before upper={          % increment counter and print number
        \stepcounter{definition}\defSquare\ \textbf{#2} \par \vspace*{2mm}
    },
    label={def:#3},         % label for referencing "def:"
    #1
}

% Example block ---------------------------------------------------------------
% Example box small icon and counter
\newcommand{\exampleSquare}{
    \setlength{\originalfboxsep}{\fboxsep}
    \setlength{\fboxsep}{2pt} % add padding
    \colorbox{iconscolor}{\textcolor{boxcolor}{\small\textbf{EXAMPLE \theexample}}}
    \setlength{\fboxsep}{\originalfboxsep} % restore \fboxsep
}

% Example block
\newcounter{example}
\makeatletter
\@addtoreset{example}{section}
\makeatother
\renewcommand{\theexample}{\thesection.\arabic{example}}
\newtcolorbox{example}[3][]{
    breakable,              % allow the box to be split across pages
    colback=boxcolor,       % background color
    colframe=boxcolor,      % frame color
    coltext=textcolor,      % text color
    boxrule=0mm,            % frame thickness
    left=0mm,               % left margin
    right=7mm,              % right margin
    top=3mm,                % top margin
    bottom=3mm,             % bottom margin
    enhanced jigsaw,        % better frame drawing
    overlay={               % overlay drawing
        \node[anchor=north west, 
              yshift=-2.5mm, 
              xshift=1mm] at (frame.north west)
              {\textcolor{iconscolor}{\faIcon{edit}}};
    },
    left=7mm,               % indent
    before upper={          % increment counter and print number
        \stepcounter{example}\exampleSquare\ \textbf{#2} \par \vspace*{2mm} 
    },
    label={ex:#3},          % label for referencing "ex:"
    #1
}

% Theorem block ----------------------------------------------------------------
% Command for the theorem box
\newcommand{\theoremSquare}{
    \setlength{\originalfboxsep}{\fboxsep} % set it to zero
    \setlength{\fboxsep}{2pt} % add padding
    \colorbox{iconscolor}{\textcolor{boxcolor}{\small\textbf{THEOREM \thetheorem}}}
    \setlength{\fboxsep}{\originalfboxsep} % restore \fboxsep
}

% Theorem block
\newcounter{theorem}
\makeatletter
\@addtoreset{theorem}{section}
\makeatother
\renewcommand{\thetheorem}{\thesection.\arabic{theorem}}
\newtcolorbox{theorem}[3][]{
    breakable,              % allow the box to be split across pages
    colback=boxcolor,       % background color
    colframe=boxcolor,      % frame color
    coltext=textcolor,      % text color
    boxrule=0mm,            % frame thickness
    left=0mm,               % left margin
    right=7mm,              % right margin
    top=3mm,                % top margin
    bottom=3mm,             % bottom margin
    enhanced jigsaw,        % better frame drawing
    overlay={               % overlay drawing
        \node[anchor=north west, 
              yshift=-2.5mm, 
              xshift=1mm] at (frame.north west)
              {\textcolor{iconscolor}{\faIcon{book}}};
    },
    left=7mm,               % indent
    before upper={          % increment counter and print number
        \stepcounter{theorem}\theoremSquare\ \textbf{#2} \par \vspace*{2mm} 
    },
    label={th:#3},          % label for referencing "th:"
    #1
}

% Corollary block --------------------------------------------------------------
% This has the same counter sas the theorem but just changes the name displayed

% Command for the corollary box
\newcommand{\corollarySquare}{
    \setlength{\originalfboxsep}{\fboxsep} % set it to zero
    \setlength{\fboxsep}{2pt} % add padding
    \colorbox{iconscolor}{\textcolor{boxcolor}{\small\textbf{COROLLARY \thetheorem}}}
    \setlength{\fboxsep}{\originalfboxsep} % restore \fboxsep
}

% Corollary block
\newtcolorbox{corollary}[3][]{
    breakable,              % allow the box to be split across pages
    colback=boxcolor,       % background color
    colframe=boxcolor,      % frame color
    coltext=textcolor,      % text color
    boxrule=0mm,            % frame thickness
    left=0mm,               % left margin
    right=7mm,              % right margin
    top=3mm,                % top margin
    bottom=3mm,             % bottom margin
    enhanced jigsaw,        % better frame drawing
    overlay={               % overlay drawing
        \node[anchor=north west, 
              yshift=-2.5mm, 
              xshift=1mm] at (frame.north west)
              {\textcolor{iconscolor}{\faIcon{book}}};
    },
    left=7mm,               % indent
    before upper={\corollarySquare\ \textbf{#2} \par \vspace*{2mm} },
    label={th:#3},          % label for referencing "th:"
    #1
}

% Proposition block ------------------------------------------------------------
% Command for the proposition box
\newcommand{\propSquare}{
    \setlength{\originalfboxsep}{\fboxsep} % set it to zero
    \setlength{\fboxsep}{2pt} % add padding
    \colorbox{iconscolor}{
        \textcolor{boxcolor}{\small\textbf{PROPOSITION \thetheorem}}
    }
    \setlength{\fboxsep}{\originalfboxsep} % restore \fboxsep
}

% Proposition block
\newtcolorbox{proposition}[3][]{
    breakable,              % allow the box to be split across pages
    colback=boxcolor,       % background color
    colframe=boxcolor,      % frame color
    coltext=textcolor,      % text color
    boxrule=0mm,            % frame thickness
    left=0mm,               % left margin
    right=7mm,              % right margin
    top=3mm,                % top margin
    bottom=3mm,             % bottom margin
    enhanced jigsaw,        % better frame drawing
    overlay={               % overlay drawing
        \node[anchor=north west, 
              yshift=-2.5mm, 
              xshift=1mm] at (frame.north west)
              {\textcolor{iconscolor}{\faIcon{book}}};
    },
    left=7mm,               % indent
    before upper={          % increment counter and print number
        \stepcounter{theorem}\propSquare\ \textbf{#2} \par \vspace*{2mm} 
    },
    label={th:#3},          % label for referencing "th:"
    #1
}

% Lemma block ------------------------------------------------------------------
% Command for the lemma box
\newcommand{\lemmaSquare}{
    \setlength{\originalfboxsep}{\fboxsep} % set it to zero
    \setlength{\fboxsep}{2pt} % add padding
    \colorbox{iconscolor}{
        \textcolor{boxcolor}{\small\textbf{LEMMA \thetheorem}}
    }
    \setlength{\fboxsep}{\originalfboxsep} % restore \fboxsep
}

% Lemma block
\newtcolorbox{lemma}[3][]{
    breakable,              % allow the box to be split across pages
    colback=boxcolor,       % background color
    colframe=boxcolor,      % frame color
    coltext=textcolor,      % text color
    boxrule=0mm,            % frame thickness
    left=0mm,               % left margin
    right=7mm,              % right margin
    top=3mm,                % top margin
    bottom=3mm,             % bottom margin
    enhanced jigsaw,        % better frame drawing
    overlay={               % overlay drawing
        \node[anchor=north west, 
              yshift=-2.5mm, 
              xshift=1mm] at (frame.north west)
              {\textcolor{iconscolor}{\faIcon{book}}};
    },
    left=7mm,               % indent
    before upper={          % increment counter and print number
        \stepcounter{theorem}\lemmaSquare\ \textbf{#2} \par \vspace*{2mm} 
    },
    label={th:#3},          % label for referencing "th:"
    #1
}