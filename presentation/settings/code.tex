%v Listings package
\usepackage{listings}

% float package to enforce figure positioning
\usepackage{float}

% New colors for syntax highlighting
\usepackage{newtxtt}

% New float package to define new floating environments
\usepackage{newfloat}

% tcolorbox for colored boxes
% [breakable] option allows the box to be split across pages
\usepackage[breakable]{tcolorbox}

% Save the original \fboxsep
\usepackage{etoolbox}

% --------------------------------- CODE ---------------------------------------

% Languages Styles -------------------------------------------------------------
% You can ADD A NEW LANGUAGE by copying and adding a new 
% \lstdefinestyle{<language>} block below

\lstdefinestyle{Pseudocode}{
    language=Python,                            % language
    basicstyle=\color{textcolor}\ttfamily\small,    % font
    commentstyle=\color{gray},                  % comments
    keywordstyle=\color{red},                   % keywords
    stringstyle=\color{azure},                  % strings
    xleftmargin=7mm,                            % left margin
    numberstyle=\color{gray},                   % line numbers
    numbers=left,                               % line numbers position
    numbersep=5mm,                              % line numbers separation
    breakatwhitespace=false,                    % break at any whitespace
    breaklines=true,                            % break lines
    captionpos=b,                               % caption position
    keepspaces=true,                            % keep spaces
    showspaces=false,                           % show spaces
    showstringspaces=false,                     % show string spaces
    showtabs=false,                             % show tabs
    tabsize=2,                                  % tab size
    morekeywords={                              % additional keywords
        function, if, else, return, 
        do, then, elif, end, and, not, 
        downto, to, exchange, each, 
        until, repeat, with
    },
    mathescape=true,                            % enable math mode
    escapeinside={§}{§}, % everithing between § characters is normal text
    comment=[l]{//},                            % line comment
    moredelim=[is][\color{blue}]{(*}{*)}, % everything between (* *) is blue
    moredelim=[is][\color{purple}]{[*}{*]} % everything between [* *] is yellow
    % literate={+}{{{\color{textcolor}+}}}{1}     % literate colors (optional)
    %          {-}{{{\color{textcolor}-}}}{1}
    %          {*}{{{\color{textcolor}*}}}{1}
    %          {>}{{{\color{textcolor}>}}}{1}
    %          {:}{{{\color{textcolor}:}}}{1}
    %          {;}{{{\color{textcolor};}}}{1}
    %          {?}{{{\color{textcolor}?}}}{1}
    %          {!}{{{\color{textcolor}!}}}{1}
    %          {=}{{{\color{textcolor}=}}}{1}
    %          {<}{{{\color{textcolor}<}}}{1}
    %          {,}{{{\color{textcolor},}}}{1}
    %          {.}{{{\color{textcolor}.}}}{1}
}

\lstdefinestyle{Python}{
    language=Python,
    basicstyle=\color{textcolor}\ttfamily\small,
    commentstyle=\color{gray},
    keywordstyle=\color{red},
    stringstyle=\color{azure},
    xleftmargin=5mm,
    breakatwhitespace=false,         
    breaklines=true,                 
    captionpos=b,                    
    keepspaces=true,                 
    showspaces=false,                
    showstringspaces=false,
    showtabs=false,                  
    tabsize=2
}

\lstdefinestyle{C++}{
    language=C++,
    basicstyle=\color{textcolor}\ttfamily\small,
    commentstyle=\color{gray},
    keywordstyle=\color{red},
    stringstyle=\color{azure},
    xleftmargin=5mm,
    breakatwhitespace=false,         
    breaklines=true,                 
    captionpos=b,                    
    keepspaces=true,                 
    showspaces=false,                
    showstringspaces=false,
    showtabs=false,                  
    tabsize=2
}

\lstdefinestyle{C}{
    language=C,
    basicstyle=\color{textcolor}\ttfamily\small,
    commentstyle=\color{gray},
    keywordstyle=\color{red},
    stringstyle=\color{azure},
    xleftmargin=5mm,                            % left margin
    numberstyle=\color{gray},                   % line numbers
    numbers=left,                               % line numbers position
    numbersep=5mm,                              % line numbers separation
    breakatwhitespace=false,                    % break at any whitespace
    breaklines=true,                            % break lines
    captionpos=b,                               % caption position
    keepspaces=true,                            % keep spaces
    showspaces=false,                           % show spaces
    showstringspaces=false,                     % show string spaces
    showtabs=false,                             % show tabs
    tabsize=2,                                  % tab size
    mathescape=true,                            % enable math mode
    escapeinside={§}{§}, % everithing between § characters is normal text
    moredelim=[is][\color{blue}]{(*}{*)}, % everything between (* *) is blue
    moredelim=[is][\color{purple}]{[*}{*]} % everything between [* *] is yellow
}

\lstdefinestyle{C2}{
    language=C,
    basicstyle=\color{textcolor}\ttfamily\fontsize{8pt}{8pt}\selectfont,
    commentstyle=\color{gray},
    keywordstyle=\color{red},
    stringstyle=\color{azure},
    xleftmargin=5mm,                            % left margin
    numberstyle=\color{gray},                   % line numbers
    numbers=left,                               % line numbers position
    numbersep=5mm,                              % line numbers separation
    breakatwhitespace=false,                    % break at any whitespace
    breaklines=true,                            % break lines
    captionpos=b,                               % caption position
    keepspaces=true,                            % keep spaces
    showspaces=false,                           % show spaces
    showstringspaces=false,                     % show string spaces
    showtabs=false,                             % show tabs
    tabsize=2,                                  % tab size
    mathescape=true,                            % enable math mode
    escapeinside={§}{§}, % everithing between § characters is normal text
    moredelim=[is][\color{blue}]{(*}{*)}, % everything between (* *) is blue
    moredelim=[is][\color{purple}]{[*}{*]}, % everything between [* *] is yellow
    moredelim=[is][\color{red}]{<*}{*>}, % everything between < > is red
}

% Code-Blocks ------------------------------------------------------------------
% New environment for code boxes, with separate counter and caption
% Usage: \begin{codeblock}...\end{codeblock} (usually with Code-Boxes inside)
\DeclareFloatingEnvironment[
    fileext=loc,        
    listname={List of Codes},
    name=Code,          % caption name
    placement=H,        % H: here, t: top, b: bottom
    within=section,     % section counter reset
]{codeblock}            % \begin{codeblock} ... \end{codeblock}

% Code -------------------------------------------------------------------------
% Colored color boxes for code, without captions
% Usage: \begin{code}{<language>}{<title>}{<input>}{<output>}...\end{code}
\tcbuselibrary{listings, skins}
\newtcblisting{code}[5][]{
    enhanced,                           % enhanced mode
    listing only,                       % only listing
    listing options={style=#2,          % language style (first argument)
                     aboveskip=-1.5mm,  % adjust top spacing here
                     belowskip=-0.5mm}, % adjust bottom spacing here
    colback=boxcolor,                   % background color
    colframe=boxcolor,                  % frame color
    boxrule=0pt,                        % frame thickness
    arc=0mm,                            % corner radius,
    #1                                  % other options (optional arguments)
    title=\small#3,                     % title (second argument)
    coltitle=textcolor,                 % title color
    fonttitle=\bfseries,                % title font
    attach boxed title to top left={    % title position
        yshift=-0.7mm,
        xshift=0mm
    },
    boxed title style={                 % title style
        colback=boxcolor, 
        boxrule=0pt,
        bottomrule=1pt
    },
    underlay unbroken={                 % title box
        \fill[boxcolor] (title.north east) rectangle (frame.north east);
    },
    underlay unbroken and first={       % top title line
        \draw[linescolor, line width=0.5pt] 
        (title.north west)-- ($(title.north west-|frame.east)$);
    },
    underlay unbroken and first={       % bottom title line
        \draw[linescolor,line width=0.5pt] 
        (frame.north west)-- (frame.north east);
    },
    underlay unbroken and last={        % bottom code-box line
        \draw[linescolor,line width=0.5pt] 
        (frame.south west)-- (frame.south east);
    },
    underlay unbroken and last={        % right side line
        \draw[linescolor,line width=0.5pt] 
        ($(title.north west-|frame.east)$)-- (frame.south east);
    },
    underlay unbroken and last={        % left side line
        \draw[linescolor,line width=0.5pt] 
        (title.north west)-- (frame.south west);
    },
    before upper={                      % print INPUT and OUTPUT
        \ifstrempty{#4}{}{
            \textcolor{gray}{\texttt{INPUT:}} \textcolor{gray}{#4}\\
        } 
        \ifstrempty{#5}{}{
            \textcolor{gray}{\texttt{OUTPUT:}} \textcolor{gray}{#5}\\
        }
    },
}

% Code-Only --------------------------------------------------------------------
% Only display the code without a box, without a caption
% Usage: \begin{codeonly}{<language>}...\end{codeonly}
\lstnewenvironment{codeonly}[2]{\lstset{style=#2}}{}

% Inline Code ------------------------------------------------------------------
% Markdown-style inline code with a colored background
\newcommand{\cc}[1]{
    \kern-1ex                       % negative space
    \tcbox[colback=boxcolor,        % background color
           coltext=textcolor!60,    % text color
           colframe=boxcolor,       % frame color   
           on line,                 % inline
           boxrule=0pt,             % frame thickness
           boxsep=0.5pt,            % frame separation
           top=0.5mm,               % top margin
           bottom=0.5mm,            % bottom margin
           left=0.5mm,              % left margin
           right=0.5mm]             % right margin
    {\texttt{#1}}                   % code text
    \kern-1ex                       % negative space
}

% Code-Boxes -------------------------------------------------------------------
% Creates a clean, pure code, separate code box
\usepackage{tcolorbox}
\tcbuselibrary{listings, skins}
\newtcblisting{codebox}[2][]{
    enhanced,                           % enhanced mode
    listing only,                       % only listing
    listing options={style=#2,          % language style (first argument)
                     aboveskip=-0.5mm,  % adjust top spacing here
                     belowskip=-0.5mm,  % adjust bottom spacing here
                     numbers=none},     % no line numbers
    colback=boxcolor,                   % background color
    colframe=boxcolor,                  % frame color
    boxrule=0pt,                        % frame thickness
    arc=3mm,                            % corner radius
    % width=\textwidth-2cm,               % box width
    #1                                  % other options (optional arguments)
}
