\documentclass[border=10pt, multi, tikz]{standalone}
\usepackage{import}
\subimport{../settings/}{init}
\usetikzlibrary{positioning}
\usetikzlibrary{calc}
\usetikzlibrary{positioning,3d}

% Colors -----------------------------------------------------------------------

% Background and boxes											% --HEX--
\definecolor{background}{HTML}{1A1B26}			% #1A1B26
\definecolor{boxcolor}{HTML}{24283B}				% #24283B
\definecolor{commentscolor}{HTML}{565F89}		% #565F89
\definecolor{darkred}{HTML}{D72323}					% #D72323
\definecolor{darkblue}{HTML}{0A84FF}				% #0A84FF
\definecolor{darkgreen}{HTML}{00C853}				% #00C853
\definecolor{darkyellow}{HTML}{FFCB30}			% #FFCB30

% Text shades																% --HEX--				
\definecolor{black}{HTML}{000000}						% #000000
\definecolor{darkgray}{HTML}{161B22}				% #161B22
\definecolor{gray}{HTML}{89929B}						% #89929B
\definecolor{lightgray}{HTML}{C6CDD5}				% #C6CDD5
\definecolor{white}{HTML}{FFFFFF}						% #FFFFFF
\definecolor{textcolor}{HTML}{FFFFFF}       % #FFFFFF

% Colors                                    % --HEX--
\definecolor{red}{HTML}{F7768E}             % #F7768E
\definecolor{orange}{HTML}{FF9E64}          % #FF9E64
\definecolor{yellow}{HTML}{FFCB30}          % #FFCB30
\definecolor{green}{HTML}{9ECE6A}           % #9ECE6A
\definecolor{azure}{HTML}{2AC3DE}           % #2AC3DE
\definecolor{blue}{HTML}{7AA2F7}            % #7AA2F7
\definecolor{purple}{HTML}{BB9AF7}          % #BB9AF7

% Background and text color
\pagecolor{background}                      % Page Background
\color{textcolor}                           % Main text color
\colorlet{captionscolor}{gray}							% Caption Colors
\colorlet{iconscolor}{white}								% Icons color
\colorlet{linescolor}{gray}									% Lines color
\colorlet{numerscolor}{commentscolor}				% Line Numbers color

% Other colors
\definecolor{rrr}{HTML}{FF0000}
\definecolor{ggg}{HTML}{00FF00}
\definecolor{bbb}{HTML}{0000FF}

\begin{document}
    \begin{tikzpicture}[scale=0.6, 3d view={120}{20}]

        % Input layer (128x128x3)
        \draw[fill=cyan] (0, 0, 0) -- (0, 0, 4) -- (0, 5, 4) -- (0, 5, 0) -- cycle;
        \node[up right, rotate=43] at (0, 0, 7.5) {\scriptsize Input 128x128x3};

        % Convolutional layer 1 (kernel 9x9, 96 channels, stride 4)
        \draw[fill=orange] (1.3, 0, 0) -- (1.3, 0, 3) -- (1.3, 3, 3) -- (1.3, 3, 0) -- cycle;
        \draw[fill=orange] (1.3, 0, 0) -- (1.3, 0, 3) -- (1.6, 0, 3) -- (1.6, 0, 0) -- cycle;
        \draw[fill=orange] (1.3, 3, 0) -- (1.3, 3, 3) -- (1.6, 3, 3) -- (1.6, 3, 0) -- cycle;
        \draw[fill=orange] (1.6, 0, 0) -- (1.6, 0, 3) -- (1.6, 3, 3) -- (1.6, 3, 0) -- cycle;
        \draw[fill=orange] (1.3, 0, 3) -- (1.3, 3, 3) -- (1.6, 3, 3) -- (1.6, 0, 3) -- cycle;
        \node[up right, rotate=43] at (1.3, 0, 7) {\scriptsize Conv 9x9, stride 4};
        \node[up right, rotate=70] at (0.9, 3, -3) {\scriptsize 96 channels};

        % Batch normalization
        %\coordinate (A) at (2, 0, 0);
        %\node[up right, rotate=43] at (2, 0, 7) {\scriptsize Batch Normalization};

        % Max pooling
        \coordinate (A) at (2.2, 0, 0);
        \coordinate (B) at (2.2, 0, 3);
        \draw[blue] (A) -- (B) -- cycle;
        \node[up right, rotate=43] at (2.2, 0, 7) {\scriptsize Max pooling};

        % Convolutional layer 2 (kernel 5x5, 256 channels, stride 1)
        \draw[fill=orange] (3.1, 0, 0) -- (3.1, 0, 2.5) -- (3.1, 2, 2.5) -- (3.1, 2, 0) -- cycle;
        \draw[fill=orange] (3.1, 0, 0) -- (3.1, 0, 2.5) -- (3.8, 0, 2.5) -- (3.8, 0, 0) -- cycle;
        \draw[fill=orange] (3.1, 2, 0) -- (3.1, 2, 2.5) -- (3.8, 2, 2.5) -- (3.8, 2, 0) -- cycle;
        \draw[fill=orange] (3.8, 0, 0) -- (3.8, 0, 2.5) -- (3.8, 2, 2.5) -- (3.8, 2, 0) -- cycle;
        \draw[fill=orange] (3.1, 0, 2.5) -- (3.1, 2, 2.5) -- (3.8, 2, 2.5) -- (3.8, 0, 2.5) -- cycle;
        \node[up right, rotate=43] at (3.1, 0, 7) {\scriptsize Conv 5x5, stride 1};
        \node[up right, rotate=70] at (3, 2.3, -3) {\scriptsize 256 channels};

        % Batch normalization
        %\coordinate (A) at (4.4, 0, 0);
        %\coordinate (B) at (4.4, 0, 2.5);
        %\draw[red, dashed] (A) -- (B) -- cycle;
        %\node[up right, rotate=43] at (4.4, 0, 7) %{\scriptsize Batch Normalization};

        % Max pooling
        \coordinate (A) at (4.4, 0, 0);
        \coordinate (B) at (4.4, 0, 2.5);
        \draw[blue] (A) -- (B) -- cycle;
        \node[up right, rotate=43] at (4.4, 0, 7) {\scriptsize Max pooling};

        % Convolutional layer 3 (kernel 3x3, 384 channels, stride 1)
        \draw[fill=orange] (5.2, 0, 0) -- (5.2, 0, 2) -- (5.2, 1, 2) -- (5.2, 1, 0) -- cycle;
        \draw[fill=orange] (5.2, 0, 0) -- (5.2, 0, 2) -- (6.3, 0, 2) -- (6.3, 0, 0) -- cycle;
        \draw[fill=orange] (5.2, 1, 0) -- (5.2, 1, 2) -- (6.3, 1, 2) -- (6.3, 1, 0) -- cycle;
        \draw[fill=orange] (6.3, 0, 0) -- (6.3, 0, 2) -- (6.3, 1, 2) -- (6.3, 1, 0) -- cycle;
        \draw[fill=orange] (5.2, 0, 2) -- (5.2, 1, 2) -- (6.3, 1, 2) -- (6.3, 0, 2) -- cycle;
        \node[up right, rotate=43] at (6, 0, 7) {\scriptsize Conv 3x3, stride 1};
        \node[up right, rotate=70] at (5, 1.2, -3) {\scriptsize 384 channels};

        % Convolutional layer 4 (kernel 3x3, 384 channels, stride 1)
        \draw[fill=orange] (7, 0, 0) -- (7, 0, 2) -- (7, 1, 2) -- (7, 1, 0) -- cycle;
        \draw[fill=orange] (7, 0, 0) -- (7, 0, 2) -- (8.1, 0, 2) -- (8.1, 0, 0) -- cycle;
        \draw[fill=orange] (7, 1, 0) -- (7, 1, 2) -- (8.1, 1, 2) -- (8.1, 1, 0) -- cycle;
        \draw[fill=orange] (8.1, 0, 0) -- (8.1, 0, 2) -- (8.1, 1, 2) -- (8.1, 1, 0) -- cycle;
        \draw[fill=orange] (7, 0, 2) -- (7, 1, 2) -- (8.1, 1, 2) -- (8.1, 0, 2) -- cycle;
        \node[up right, rotate=43] at (7.6, 0, 7) {\scriptsize Conv 3x3, stride 1};
        \node[up right, rotate=70] at (6.5, 1.2, -3) {\scriptsize 384 channels};

        % Batch normalization
        %\coordinate (A) at (7.2, 0, 0);
        %\coordinate (B) at (7.2, 0, 2);
        %\draw[red, dashed] (A) -- (B) -- cycle;
        %\node[up right, rotate=43] at (7.2, 0, 7) %{\scriptsize Batch Normalization};

        % Convolutional layer 5 (kernel 3x3, 256 channels, stride 1)
        \draw[fill=orange] (8.9, 0, 0) -- (8.9, 0, 2) -- (8.9, 1, 2) -- (8.9, 1, 0) -- cycle;
        \draw[fill=orange] (8.9, 0, 0) -- (8.9, 0, 2) -- (9.6, 0, 2) -- (9.6, 0, 0) -- cycle;
        \draw[fill=orange] (8.9, 1, 0) -- (8.9, 1, 2) -- (9.6, 1, 2) -- (9.6, 1, 0) -- cycle;
        \draw[fill=orange] (9.6, 0, 0) -- (9.6, 0, 2) -- (9.6, 1, 2) -- (9.6, 1, 0) -- cycle;
        \draw[fill=orange] (8.9, 0, 2) -- (8.9, 1, 2) -- (9.6, 1, 2) -- (9.6, 0, 2) -- cycle;
        \node[up right, rotate=43] at (9.1, 0, 7) {\scriptsize Conv 3x3, stride 1};
        \node[up right, rotate=70] at (8.7, 1.4, -3) {\scriptsize 256 channels};

        % Batch normalization
        %\coordinate (A) at (9.1, 0, 0);
        %\coordinate (B) at (9.1, 0, 2);
        %\draw[red, dashed] (A) -- (B) -- cycle;
        %\node[up right, rotate=43] at (9.1, 0, 7) %{\scriptsize Batch Normalization};

        % Max pooling
        \coordinate (A) at (10, 0, 0);
        \coordinate (B) at (10, 0, 2);
        \draw[blue] (A) -- (B) -- cycle;
        \node[up right, rotate=43] at (10, 0, 7) {\scriptsize Max pooling};

        % Fully connected layer 1 (1024 -> 512 features)
        \draw[fill=green] (10.7, -0.3, 0) -- (11.0, -0.3, 0) -- (11.0, 5, 0) -- (10.7, 5, 0) -- cycle;
        \node[up right, rotate=43] at (11, 0, 7) {\scriptsize Fully connected 1024 -> 512};

        % Fully connected layer 2 (512 -> 512 features)
        \draw[fill=green] (11.8, -0.3, 0) -- (11.5, -0.3, 0) -- (11.5, 4, 0) -- (11.8, 4, 0) -- cycle;
        \node[up right, rotate=43] at (11.6, -0.5, 5) {\scriptsize Fully connected 512 -> 512};

        % Dropout
        %\coordinate (A) at (11.1, -0.3, 0);
        %\coordinate (B) at (11.1, 5, 0);
        %\draw[yellow, dashed] (A) -- (B) -- cycle;
        %\node[up right, rotate=43] at (11.5, 0, 5) %{\scriptsize Dropout};

        % Fully connected layer 3 (512 -> 128 features)
        \draw[fill=green] (12.4, -0.3, 0) -- (12.7, -0.3, 0) -- (12.7, 3.5, 0) -- (12.4, 3.5, 0) -- cycle;
        \node[up right, rotate=43] at (13, 0, 7) {\scriptsize Fully connected 512 -> 128};

        % Dropout
        %\coordinate (A) at (12.5, -0.3, 0);
        %\coordinate (B) at (12.5, 3.5, 0);
        %\draw[yellow, dashed] (A) -- (B) -- cycle;
        %\node[up right, rotate=43] at (12.9, 0, 5) %{\scriptsize Dropout};

        % Output layer (128 features -> 4 classes)
        \draw[fill=blue] (13.2, -0.3, 0) -- (13.5, -0.3, 0) -- (13.5, 2, 0) -- (13.2, 2, 0) -- cycle;
        \node[up right, rotate=43] at (13.9, 0, 5) {\scriptsize Output 128 -> 4};
        
        \end{tikzpicture}
\end{document}
