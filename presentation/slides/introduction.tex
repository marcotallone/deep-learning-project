\documentclass[../presentation.tex]{subfiles} % Parent file
\graphicspath{{\subfix{../images/}}} % Images path

\begin{document}

\section{Introduction} % This if you want to set the gray section title

% Slide 1 ──────────────────────────────────────────────────────────────────────
\begin{frame}

	\frametitle{Introduction}

	Hello, world! \emoji{smile}

	\begin{cbox}
		Hellow world in a box.
	\end{cbox}

	\begin{cbox}[orange!50!background]
		Hello world in an orange box mixed $50\%$ with the background.
	\end{cbox}

	\begin{warning}
		Recall you need lualatex to compile this template.
	\end{warning}

	\begin{example}[Title of the example block]
		For example like this \faIcon{smile}.
	\end{example}

\end{frame}

% Slide 2 ──────────────────────────────────────────────────────────────────────
\begin{frame}

	\frametitle{Example TikZ Picture}

	Minimal \bft{working} example of a \itt{TikZ picture}.

	\begin{center}
		\begin{tikzpicture}

			% Figures with text inside
			\node[draw, circle, minimum size=1cm] (dumb-name) at (0,0) {A};

			% Figures withouth text
			\draw[
				fill=red, % Set the color of the shape
				fill opacity=0.5, % Set the opacity of the shape
				draw=blue, % Set the border color
				draw opacity=1, % Set the border opacity
				line width=1mm % Set the border width
			] (2,0) circle (1cm);

			% Arrows and segments with coordinates
			\coordinate (A) at (0,2);
			\coordinate (B) at (2,3);
			\coordinate (C) at (3.5,2.5);
			\coordinate (D) at (4,2);
			\coordinate (E) at (2,1.5);

			\draw[-Stealth] (A) -- (B) -- (C) -- (D); % Direct piecewise path
			
			% Multiple paths
			\draw[green] (A) -- (C);
			\draw[dashed, yellow] (C) -- (E);
			\draw[-Stealth, red] (E) -- (D);

		\end{tikzpicture}
	\end{center}

\end{frame}

\end{document}